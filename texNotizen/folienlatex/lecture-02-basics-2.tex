\errorcontextlines=5

%%%

\documentclass[
%handout,
]{beamer}

\usepackage{ifluatex}
\ifluatex\else\errmessage{This document requires LuaLaTeX}\fi

\usepackage{etex,etoolbox}
\usepackage{fontspec}
\usepackage[ngerman]{babel}
\usepackage{csquotes}
\usepackage{array}
\usepackage{wrapfig}
\usepackage{booktabs}
\usepackage{ccicons}
\usepackage{calc}

\usepackage{tikz}
\usetikzlibrary{arrows,intersections,calc,through,%
  external,positioning,automata,datavisualization,%
  datavisualization.formats.functions}

\usepackage{luacode}
\usepackage{pgfplots}
\usepackage{manfnt}

%%% title and such

\title{Wissenschaftliches Arbeiten mit \LaTeX}
\author{\texorpdfstring{Felix Hilsky\\basierend auf einem Kurs von\\Daniel Borchmann,\\Tom Hanika und \\Max Marx}{Felix Hilsky basierend auf einem Kurs von Daniel Borchmann, Tom Hanika und Max Marx}}
\titlegraphic{\ccLogo \ccAttribution \ccShareAlike}

%%% theme

\usepackage{tikz}
\usetikzlibrary{shapes.multipart}
\usetheme{CambridgeUS}
\setbeamertemplate{blocks}[rounded][shadow=false]
\setbeamertemplate{items}{\raisebox{0.3ex}{%
    \tikz[scale=0.13] \draw[fill] (0,0) -- (0,1) -- (0.9,0.5) -- cycle;}}
\usetikzlibrary{arrows}
\tikzset{>={stealth'[sep]}}
\setbeamertemplate{navigation symbols}{}
\setbeamertemplate{footline}{}
\setlength{\abovedisplayskip}{0pt}
\setbeamerfont{title}{series=\bfseries}
\defbeamertemplate{block alerted begin}{bends}{%
  \begin{columns}
    \begin{column}{0.05\linewidth}
      \dbend
    \end{column}
    \begin{column}{0.95\linewidth}
      \vskip.75ex\usebeamercolor[fg]{block title
        alerted}\insertblocktitle{}
      \vskip.1em
      \usebeamercolor[fg]{normal text}
}
\defbeamertemplate{block alerted end}{bends}{%
    \end{column}
  \end{columns}
}
%%%

\mode<handout>{
  \usepackage{pgfpages}
  \pgfpagesuselayout{2 on 1}[a4paper,border shrink=5mm]
}

%%% lecture organization

\usepackage{xparse}
\DeclareDocumentCommand \Lecture { m m }{%
  \lecture{#1}{#2}
  \part{#1}
  \include{#2}
}

\AtBeginSection{
  \setbeamertemplate{blocks}[rounded][shadow=true]
  \begin{frame}[plain]
    \begin{block}{}
      \begin{center}
        \textcolor{darkred}{\textbf{\Large \strut\smash{\insertpart}}}\\[1ex]
        \textcolor{blue!70!black}{\strut\smash{\insertsection}}
      \end{center}
    \end{block}
  \end{frame}
  \setbeamertemplate{blocks}[rounded][shadow=false]
  \setbeamertemplate{block alerted begin}[bends]
  \setbeamertemplate{block alerted end}[bends]
}

%%% misc

\newcommand{\GNULinux}{GNU\lower-0.25ex\hbox{/}Linux}
\newcommand{\TikZ}{Ti\emph{k}Z}

\usepackage{listings}

\lstset{language=[LaTeX]TeX, basicstyle=\ttfamily,
  keywordstyle={\color{blue}\bfseries}, frame=tb, extendedchars=true, literate=%
  {ä}{{\"a}}1 {ö}{{\"o}}1, escapeinside={(*@}{@*)}, mathescape=true,
  basewidth=0.5em, keywordstyle={\color{blue}},
  morekeywords={[0]includegraphics,rotatebox,scalebox,resizebox,providecommand,
    subsection,subsubsection,paragraph,subparagraph,part,chapter,tableofcontents,
    mathring,text,mathbb,printindex,addbibresource,printbibliography,subtitle,
    institute,titlegraphic,subject,keywords,draw,path,color,textcolor,toprule,
    midrule,bottomrule,maketitle,setlength,enquote,listoffigures,listoftables,
    theoremstyle,theoremheaderfont,theorembodyfont,newblock,parencite,footcite,
    autocite,bibitem,middle,tikzset,usetikzlibrary,coordinate,node,foreach,
    datavisualization,varepsilon,autocite,bibitem,DeclareRobustCommand,
    DeclareDocumentCommand,IfBooleanTF,bye,frametitle,setbeamertemplate,pause,
    onslide,uncover,visible,invisible,only,alt,temporal,alert,AtBeginSection,
    usetheme,setbeamerfont,tikz,includeonlyframes,mode,pgfpagesuselayout,RequirePackage,
  },
}

\AtBeginDocument{\frame[plain]{\maketitle}}

%%% end of preamble
\subtitle{Abschnitte, Tabellen und Bilder}
\date{2017-05-16}

\begin{document}

\begin{frame}
  \frametitle{Ziel Heute}

  \begin{itemize}
  \item Dokumentstruktur
  \item Tabellen
  \item Bilder einbinden (nicht erstellen!)
  \item Abbildungen
  % \item Verweise innerhalb eines Dokuments
  \end{itemize}

\end{frame}

\begin{frame}[fragile]
  \frametitle{Abschnitte}

  \begin{itemize}
    \item<+-> geben die Grobstruktur des Dokuments an
    \item<+-> In \LaTeX\ mit
      \begin{itemize}
        \item \lstinline!\part!, \lstinline!\part*!
        \item \lstinline!\chapter!, \lstinline!\chapter*!
          (nicht in \lstinline!article!/ \lstinline!scrartcl!)
        \item \lstinline!\section!, \lstinline!\section*!
        \item \lstinline!\subsection!, \lstinline!\subsection*!
        \item \lstinline!\subsubsection!, \lstinline!\subsubsection*!
        \item \lstinline!\paragraph!, \lstinline!\paragraph*!
        \item \lstinline!\subparagraph!, \lstinline!\subparagraph*!
      \end{itemize}
    \item<+-> *-Formen werden nicht nummeriert und treten auch nicht im
      Inhaltsverzeichnis auf
    \item<+-> Abschnitte werden nicht explizit beendet.
\begin{lstlisting}
\section{Einführung}
Jetzt labert man ein bisschen.
\subsection*{Notation}
Notation muss halt benannt werden.
\end{lstlisting}
  \end{itemize}
%   \item<+-> Inhaltsverzeichnisse mit
% \begin{lstlisting}
% \tableofcontents
% \end{lstlisting}
%     und zweimaligem Übersetzen.
%   \end{itemize}

\end{frame}

\section{Tabellen}

\begin{frame}[fragile]
  \frametitle{Tabellen}

  \onslide<1->

  \LaTeX\ stellt die Grundfunktionalität für Tabellen bereit

  \begin{columns}
    \begin{column}{0.4\linewidth}
      \begin{block}<3->{}
\begin{lstlisting}
\begin{tabular}{lr|c||l}
  \hline
  1 & 2 & 3 & 4 \\
  \hline\hline
  5 & 6 & 7 & 8 \\
  9 & 0 & 1 & 2 \\
  \hline
\end{tabular}
\end{lstlisting}
      \end{block}
    \end{column}
    \begin{column}{0.5\linewidth}
      \centering
      \onslide<2->{%
        \begin{tabular}{lr|c||l}
          \hline
          1 & 2 & 3 & 4 \\
          \hline\hline
          5 & 6 & 7 & 8 \\
          9 & 0 & 1 & 2 \\
          \hline
        \end{tabular}
      }
    \end{column}
  \end{columns}

  \begin{itemize}
  \item<4-> \lstinline!{lr|c||l}! ist das \emph{Tabellenformat}
  \item<5-> Spalten werden mit \lstinline|&| unterteilt
  \item<6-> Zeilen werden mit \lstinline|\\| beendet
  \item<7-> \lstinline|\hline| ergibt eine horizontale Linie
  \end{itemize}

\end{frame}

\begin{frame}[fragile]
  \frametitle{Tabellen}

  \onslide<+->

  \begin{itemize}
  \item<+-> Tabellenformat
    \onslide<+->
    \begin{itemize}
    \item \lstinline|c| ergibt eine zentrierte Spalte
    \item \lstinline|r| ergibt eine rechtsbündige Spalte
    \item \lstinline|l| ergibt eine linksbündige Spalte
    \item \lstinline!p{5cm}! ergibt eine Spalte der Breite 5cm
    \item \lstinline!|! ergibt eine vertikale Linie
    \end{itemize}
  \item<+-> \lstinline!\cline{4-7}! ergibt je eine vertikale Linie von Spalte 4 bis Spalte 7
  \item<+-> \lstinline!\multicolumn{3}{|c|}{Inhalt}! formatiert die nächsten drei Spalten
    im Format \lstinline!|c|! mit \lstinline!Inhalt!
  \end{itemize}

  \onslide<+->

  \medskip

  \begin{center}
    \begin{tabular}{|lr|c|p{2cm}|}
      Hier & geht's & RUND! & \dots \\
      \hline\hline
      1    &      2 & \multicolumn{2}{c}{ DreiDreiDrei! } \\
      \cline{2-3}
      4    &      5 & 6     & 7
    \end{tabular}
  \end{center}

\end{frame}

\begin{frame}[fragile]
  \frametitle{Tabellenlayout (modern)}

  \onslide<+->

  Schönere Tabellenstriche mittels

\begin{lstlisting}
\usepackage{booktabs}
\end{lstlisting}

  \onslide<+->

  \bigskip

  Dann:

  \begin{columns}
    \begin{column}{0.55\linewidth}
\begin{lstlisting}
\begin{tabular}{l|cr}
  \toprule
  Tabelle & Kopf  & Kopf  \\
  \midrule
  Zeile   & Zelle & Zelle \\
  Zeile   & Zelle & Zelle \\
  \bottomrule
\end{tabular}
\end{lstlisting}
    \end{column}
    \begin{column}{0.45\linewidth}
      \centering
      \begin{tabular}{l|cr}
        \toprule
        Tabelle & Kopf & Kopf \\
        \midrule
        Zeile & Zelle & Zelle \\
        Zeile & Zelle & Zelle \\
        \bottomrule
      \end{tabular}
    \end{column}
  \end{columns}

\end{frame}

\begin{frame}
  \frametitle{Pakete für Tabellen}

  \onslide<+->

  Es gibt eine Reihe von nützlichen Paketen für Tabellen

  \begin{itemize}
  \item<+-> \lstinline!array! für erweiterte Tabellenformate (und kleine Korrekturen)
  \item<+-> \lstinline!tabularx! für noch mehr Tabellenformate
  \item<+-> \lstinline!longtable! für Tabellen, die über mehrere Seiten gehen
  \item<+-> \dots
  \end{itemize}

\end{frame}

\section{Bilder einbinden}

\begin{frame}[fragile]
  \frametitle{Bilder einbinden}

  \onslide<+->

  \begin{itemize}
  \item Einbinden von Graphiken in \LaTeX\ mit Hilfe des Pakets \texttt{graphicx}
  \item Befehl
\begin{lstlisting}
\includegraphics[(*@\textit{Optionen}@*)]{(*@\textit{Bildname}@*)}
\end{lstlisting}
  \end{itemize}

  \onslide<+->

  \begin{Beispiel}
\begin{lstlisting}
\centerline{\includegraphics[width=0.3\linewidth{bild.jpg}}
\end{lstlisting}% ,keepaspectratio]

    ergibt

    \centerline{\includegraphics[width=0.3\linewidth]{pics/willersuhr.jpeg}} % ,keepaspectratio
  \end{Beispiel}

\end{frame}

\begin{frame}[fragile]
  \frametitle{Optionen zum Einbinden von Graphiken}

  \onslide<+->

  Oft verwendete Optionen von \lstinline{\includegraphics} sind
  \begin{itemize}
    \item \texttt{width}, \texttt{height} für Breite und Höhe
    % \item \texttt{keepaspectratio}, wenn Breite \emph{und} Höhe angegeben, sind es beides Maximalmaße und Bild wird passend skaliert %so dass nach Angabe von Breite oder Höhe
      % das Bild automatisch skaliert wird
    \item \texttt{scale} zur Skalierung des Bildes
    \item \texttt{angle} zur Angabe eines Drehwinkels
    \item \texttt{origin} zur Angabe des Drehpunktes
  \end{itemize}

  \onslide<+->

  \begin{Beispiel}
\begin{lstlisting}
\centerline{\includegraphics[scale=1.2,origin=cc,
    angle=42]{bild.jpg}}
\end{lstlisting}

    \centerline{\includegraphics[scale=0.08,origin=cc,angle=42]{pics/willersuhr.jpeg}}
  \end{Beispiel}

\end{frame}

\begin{frame}[fragile]
  \frametitle{Bildformate}
  Erkannte Bildformate sind
  \begin{itemize}
    \item pdf
    \item png
    \item jp(e)g
    \item eps
  \end{itemize}
  Vektorgraphiken (.svg-Dateien) können in tikz-Code umgewandelt und damit in \LaTeX eingebunden werden.
  
\end{frame}
\begin{frame}[fragile]
  \frametitle{Weitere Befehle aus \texttt{graphicx}}

  \onslide<+->

  \begin{itemize}[<+->]
  \item Drehen von Inhalten mit
    \lstinline!\rotatebox[$\textit{Optionen}$]{$\textit{Winkel}$}{$\textit{Inhalt}$}!
    \onslide<+->
\begin{lstlisting}
\rotatebox[origin=lB]{-30}{TextTextTextText}
\end{lstlisting}
    \rotatebox[origin=lB]{-30}{TextTextTextText}
  \item \lstinline!\resizebox{$\textit{Breite}$}{$\textit{Höhe}$}{$\textit{Text}$}!
    \onslide<+->
\begin{lstlisting}
\resizebox{1cm}{.4cm}{Hier ist es eng...}
\end{lstlisting}
    \resizebox{1cm}{.4cm}{Hier ist es eng...}
  \item \lstinline!\scalebox{$\textit{horizontal}$}[$\textit{vertikal}$]{$\textit{Text}$}!
    \onslide<+->
\begin{lstlisting}
\scalebox{3}[-1]{Breitergehtnicht}
\end{lstlisting}
    \scalebox{3}[-1]{Breitergehtnicht}
  \end{itemize}
\end{frame}

\begin{frame}[fragile]
  \frametitle{Ausblick: Grafiken erstellen}

  \begin{center}
    \begin{tikzpicture}[
        thick,
        >=stealth',
        dot/.style = {
          draw,
          fill = white,
          circle,
          inner sep = 0pt,
          minimum size = 4pt
        }
      ]
      \coordinate (O) at (0,0);
      \draw[->] (-0.3,0) -- (8,0) coordinate[label = {below:$x$}] (xmax);
      \draw[->] (0,-0.3) -- (0,5) coordinate[label = {right:$f(x)$}] (ymax);
      \path[name path=x] (0.3,0.5) -- (6.7,4.7);
      \path[name path=y] plot[smooth] coordinates {(-0.3,2) (2,1.5) (4,2.8) (6,5)};
      \scope[name intersections = {of = x and y, name = i}]
        \fill[gray!20] (i-1) -- (i-2 |- i-1) -- (i-2) -- cycle;
        \draw      (0.3,0.5) -- (6.7,4.7) node[pos=0.8, below right] {Sekante};
        \draw[red] plot[smooth] coordinates {(-0.3,2) (2,1.5) (4,2.8) (6,5)};
        \draw (i-1) node[dot, label = {above:$P$}] (i-1) {} -- node[left,yshift=-3pt]
          {$f(x_0)$} (i-1 |- O) node[dot, label = {below:$x_0$}] {};
        \path (i-2) node[dot, label = {above:$Q$}] (i-2) {} -- (i-2 |- i-1)
          node[dot] (i-12) {};
        \draw           (i-12) -- (i-12 |- O) node[dot,
                                  label = {below:$x_0 + \varepsilon$}] {};
        \draw[blue, <->] (i-2) -- node[right] {$f(x_0 + \varepsilon) - f(x_0)$}
                                  (i-12);
        \draw[blue, <->] (i-1) -- node[below] {$\varepsilon$} (i-12);
        \path       (i-1 |- O) -- node[below] {$\varepsilon$} (i-2 |- O);
        \draw[gray]      (i-2) -- (i-2 -| xmax);
        \draw[gray, <->] ([xshift = -0.5cm]i-2 -| xmax) -- node[fill = white]
          {$f(x_0 + \varepsilon)$}  ([xshift = -0.5cm]xmax);
      \endscope
    \end{tikzpicture}
  \end{center}

  \onslide<2->{mit \textcolor{red}{Ti\textit{k}Z} $\leadsto$ später!}

  \vfill\hbox{}\hfill\hbox{\tiny\url{http://www.texample.net/tikz/examples/linear-regression/}}

\end{frame}


\section{Abbildungen}

\begin{frame}[fragile]
  \frametitle{Abbildungen}

  \begin{itemize}
  \item<+-> Größere Bilder und Tabellen werden mittels \emph{Gleitumgebungen (Floats)} gesetzt:
  \onslide<+->
\begin{lstlisting}
\begin{figure}
  $\dots$
  \caption{Bildunterschrift}
\end{figure}
\end{lstlisting}
  \onslide<+->
  \LaTeX\ platziert dann die Bilder auf der aktuellen oder auf einer der folgenden Seiten.

  \item<+-> Für Tabellen gibt es die spezielle \texttt{table}-Umgebung.
  \item<+-> Verzeichnisse für Abbildungen und Tabellen mit \lstinline!\listoffigures! und
    \lstinline!\listoftables!.
  \item<+-> Nützliches Paket: \lstinline!float!: zur Definition weiterer Gleitumgebungen.
  \end{itemize}

\end{frame}

\begin{frame}[fragile]
  \frametitle{Platzierung von Floats}

  \onslide<+->

  Die Platzierung wird durch die entsprechenden \textit{Optionen} angegeben:

  \onslide<+->

  \begin{description}
  \item[h] Platzierung an der aktuellen Stelle
  \item[t] Platzierung oben auf einer Seite
  \item[b] Platzierung unten auf einer Seite
  \item[p] Platzierung auf einer extra Seite
  \end{description}

  \onslide<+->

  Optionen können gemischt werden.

  \onslide<+->

\begin{lstlisting}
\begin{figure}[tp]
  Diese \enquote{Abbildung} erscheint entweder oben auf
  einer Seite, oder auf einer extra Seite.

  \caption{Bildunterschrift}
\end{figure}
\end{lstlisting}

\end{frame}

% \section{Referenzieren}

% \begin{frame}[fragile]
%   \frametitle{Verweise im Dokument}

%   \onslide<+->

%   \LaTeX\ erlaubt die automatische Erstellung von Verweisen innerhalb des Dokuments

%   \begin{itemize}
%   \item<+-> mit dem Befehl \lstinline!\label{label-name}! wird ein \emph{Label} im Dokument
%     gesetzt
%   \item<+-> mit dem Befehl \lstinline!\ref{label-name}! wird auf dieses Label verwiesen
%   \end{itemize}

%   \onslide<+->

% \begin{lstlisting}
% \section{Einführung}
% \label{sec:introduction}

% Das Problem, welches wir behandeln wollen, ist wichtig!

% \section{Das Problem}

% Siehe Abschnitt~\ref{sec:introduction}!
% \end{lstlisting}

%   \onslide<+->

%   \emph{Wichtig}: Zweimaliges Übersetzen notwendig!

% \end{frame}

% \begin{frame}[fragile]
%   \frametitle{Platzierung von Labeln}

%   \onslide<+->

%   Die Formatierung von \lstinline!\ref{label-name}! hängt von dem Verweis ab.

%   \onslide<+->

% \begin{lstlisting}
% \section{Abschnitt}
% \label{sec:section}         % Verweis auf Abschnittsnummer

% \begin{enumerate}
% \item\label{item:1} Eintrag % Verweis auf Einzelpunkt
% \end{enumerate}

% \begin{figure}
%   $\dots$
%   \caption{\label{figure} Bildunterschrift}
%                             % Verweis auf Abbildung
% \end{figure}
% \end{lstlisting}

%   \onslide<+->

%   Verweis auf die Seitenzahl mit \lstinline!\pageref{label-name}!.

% \end{frame}

% \begin{frame}[fragile]
%   \frametitle{Nützliche Pakete}

%   \onslide<+->

%   Es gibt einige nützliche Pakete, die Verweise besser formatieren können

%   \begin{itemize}
%   \item<+-> \lstinline!ntheorem! gibt den Befehl \lstinline!\thref{thm:main-theorem}!,
%     welcher automatisch den Typ der Aussage hinzufügt (Satz~5.1, Lemma~5.1, Bemerkung~5.1,
%     \dots)
%   \item<+-> \lstinline!cleveref! gibt \lstinline!\cref! und weitere Befehle, welche
%     automatisch den Typ der Referenz hinzufügen
%   \item<+-> \lstinline!varioref! gibt \lstinline!\vref!, \lstinline!\vpageref!, und
%     weitere, welche intelligente Formatierungen abhängig vom Abstand zwischen Referenz und
%     Verweis erlauben
%   \end{itemize}

% \end{frame}

\end{document}

%%% Local Variables:
%%% mode: latex
%%% TeX-master: t
%%% TeX-engine: luatex
%%% ispell-local-dictionary: "de_DE"
%%% End:
