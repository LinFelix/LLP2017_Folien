\errorcontextlines=5

%%%

\documentclass[
%handout,
]{beamer}

\usepackage{ifluatex}
\ifluatex\else\errmessage{This document requires LuaLaTeX}\fi

\usepackage{etex,etoolbox}
\usepackage{fontspec}
\usepackage[ngerman]{babel}
\usepackage{csquotes}
\usepackage{array}
\usepackage{wrapfig}
\usepackage{booktabs}
\usepackage{ccicons}
\usepackage{calc}

\usepackage{tikz}
\usetikzlibrary{arrows,intersections,calc,through,%
  external,positioning,automata,datavisualization,%
  datavisualization.formats.functions}

\usepackage{luacode}
\usepackage{pgfplots}
\usepackage{manfnt}

%%% title and such

\title{Wissenschaftliches Arbeiten mit \LaTeX}
\author{\texorpdfstring{Felix Hilsky\\basierend auf einem Kurs von\\Daniel Borchmann,\\Tom Hanika und \\Max Marx}{Felix Hilsky basierend auf einem Kurs von Daniel Borchmann, Tom Hanika und Max Marx}}
\titlegraphic{\ccLogo \ccAttribution \ccShareAlike}

%%% theme

\usepackage{tikz}
\usetikzlibrary{shapes.multipart}
\usetheme{CambridgeUS}
\setbeamertemplate{blocks}[rounded][shadow=false]
\setbeamertemplate{items}{\raisebox{0.3ex}{%
    \tikz[scale=0.13] \draw[fill] (0,0) -- (0,1) -- (0.9,0.5) -- cycle;}}
\usetikzlibrary{arrows}
\tikzset{>={stealth'[sep]}}
\setbeamertemplate{navigation symbols}{}
\setbeamertemplate{footline}{}
\setlength{\abovedisplayskip}{0pt}
\setbeamerfont{title}{series=\bfseries}
\defbeamertemplate{block alerted begin}{bends}{%
  \begin{columns}
    \begin{column}{0.05\linewidth}
      \dbend
    \end{column}
    \begin{column}{0.95\linewidth}
      \vskip.75ex\usebeamercolor[fg]{block title
        alerted}\insertblocktitle{}
      \vskip.1em
      \usebeamercolor[fg]{normal text}
}
\defbeamertemplate{block alerted end}{bends}{%
    \end{column}
  \end{columns}
}
%%%

\mode<handout>{
  \usepackage{pgfpages}
  \pgfpagesuselayout{2 on 1}[a4paper,border shrink=5mm]
}

%%% lecture organization

\usepackage{xparse}
\DeclareDocumentCommand \Lecture { m m }{%
  \lecture{#1}{#2}
  \part{#1}
  \include{#2}
}

\AtBeginSection{
  \setbeamertemplate{blocks}[rounded][shadow=true]
  \begin{frame}[plain]
    \begin{block}{}
      \begin{center}
        \textcolor{darkred}{\textbf{\Large \strut\smash{\insertpart}}}\\[1ex]
        \textcolor{blue!70!black}{\strut\smash{\insertsection}}
      \end{center}
    \end{block}
  \end{frame}
  \setbeamertemplate{blocks}[rounded][shadow=false]
  \setbeamertemplate{block alerted begin}[bends]
  \setbeamertemplate{block alerted end}[bends]
}

%%% misc

\newcommand{\GNULinux}{GNU\lower-0.25ex\hbox{/}Linux}
\newcommand{\TikZ}{Ti\emph{k}Z}

\usepackage{listings}

\lstset{language=[LaTeX]TeX, basicstyle=\ttfamily,
  keywordstyle={\color{blue}\bfseries}, frame=tb, extendedchars=true, literate=%
  {ä}{{\"a}}1 {ö}{{\"o}}1, escapeinside={(*@}{@*)}, mathescape=true,
  basewidth=0.5em, keywordstyle={\color{blue}},
  morekeywords={[0]includegraphics,rotatebox,scalebox,resizebox,providecommand,
    subsection,subsubsection,paragraph,subparagraph,part,chapter,tableofcontents,
    mathring,text,mathbb,printindex,addbibresource,printbibliography,subtitle,
    institute,titlegraphic,subject,keywords,draw,path,color,textcolor,toprule,
    midrule,bottomrule,maketitle,setlength,enquote,listoffigures,listoftables,
    theoremstyle,theoremheaderfont,theorembodyfont,newblock,parencite,footcite,
    autocite,bibitem,middle,tikzset,usetikzlibrary,coordinate,node,foreach,
    datavisualization,varepsilon,autocite,bibitem,DeclareRobustCommand,
    DeclareDocumentCommand,IfBooleanTF,bye,frametitle,setbeamertemplate,pause,
    onslide,uncover,visible,invisible,only,alt,temporal,alert,AtBeginSection,
    usetheme,setbeamerfont,tikz,includeonlyframes,mode,pgfpagesuselayout,RequirePackage,
  },
}

\AtBeginDocument{\frame[plain]{\maketitle}}

%%% end of preamble
\subtitle{Setzen Mathematischer Formeln}
\date{2017-05-02}

\begin{document}

\begin{frame}
  \frametitle{Ziele dieses Abschnitts (oder: Wozu \TeX\ geschaffen wurde)}

  \onslide<+->

  \begin{equation*}
    \frac{f\left(\zeta\right)}{\zeta-z_0} = \frac{f\left(\zeta\right)}
    {\zeta-z_0}\frac{1}{
      1-\frac{z-z_0}{\zeta-z_0}} = \sum_{n=0}^{\infty}\frac{f\left(\zeta\right)}
    {\zeta-z_0}
    \left(\frac{z-z_0}{\zeta-z_0}\right)^n
  \end{equation*}

  \medskip

  \begin{equation*}
    0 \neq \left|\, \frac{1}{10^{10}} \left( \sum_{n = -\infty}^{\infty}
        e^{\frac{n^2}{10^{10}}} \right)^2 - \pi \,\right|
    \le 10^{-42 \cdot 10^9}
  \end{equation*}

  \medskip

  \begin{equation*}
    \frac{1}{\pi} = \frac{2\sqrt{2}}{9801} \sum^\infty_{k=0} \frac{(4k)!(1103+26390k)}{(k!)^4 396^{4k}}
  \end{equation*}

\end{frame}

\begin{frame}
  \frametitle{Setzen mathematischer Formeln}

  \onslide<+->

  Prinzipiell ist Mathematik eine ganz andere Welt (nicht nur in \LaTeX), denn
  \begin{itemize}
  \item<+-> das Setzen mathematischer Formeln erfolgt in einer eigenen Umgebung
  \item<+-> mit eigenen Befehlen,
  \item<+-> eigener Schrift,
  \item<+-> eigenen Einstellungen
  \item<+-> und eigenen Tücken \dots
  \end{itemize}

  \onslide<+->

  aber es ist einfacher (und schöner) als in den meisten (allen) anderen Textsatzsystemen.
\end{frame}

%%% HERE

\section{Grundlagen}

\begin{frame}[fragile]
  \frametitle{Das Grundgerüst}

  % \onslide<+->

  \begin{block}{Zwei grundlegende Modi für Mathematik}
    \begin{itemize}
    \item<+-> Im laufenden Text mit
      \begin{itemize}
        \item \lstinline!$\text{\$}\dots\text{\$}$!
        \item \lstinline!\($\dots$\)!
        \item \lstinline!\begin{math}$\dots$\end{math}!
      \end{itemize}
      Wähle eines und nutze es konsistent!
    \item<+-> oder abgesetzt mit
      \begin{itemize}
        \item \lstinline!\[$\dots$\]!
        \item \lstinline!\begin{displaymath}$\dots$\end{displaymath}!
        \item \textbf{Nicht:} \lstinline!$\text{\$\$}\dots\text{\$\$}$! (Veraltet!)
        \item \lstinline!\begin{equation}$\dots$\end{equation}! (Empfehlung für einzeilige Formeln)
        \item \lstinline!\begin{align}$\dots$\end{align}! (Empfehlung für mehrzeilige Formeln)
      \end{itemize}
    \end{itemize}
  \end{block}
\end{frame}

\begin{frame}[fragile]
  \frametitle{Formelelemente}
    \begin{itemize}
    \item Buchstaben, dargestellt als \textit{jeweils ein} Symbol, $xyz$,
    \item Zahlen: $123$,
    \item griechische Buchstaben $\gamma, \varepsilon, \xi, \ldots$ und
      hebräische wie $\aleph, \beth, \ldots$
    \item Operationen: $+$,$-$ und $\cdot$ (mit \lstinline!\cdot!)
    \item Sub- und Superskripte: $\text{\lstinline!x^2!}  \mathrel{\hat=} x^2$
      und $\text{\lstinline!x_2!} \mathrel{\hat=} x_2$,
    \item Brüche: \lstinline!\frac{$\textit{Zähler}$}{$\textit{Nenner}$}!
      ${} \mathrel{\hat=} \frac{1}{n}$,
    \item Wurzeln: \lstinline!\sqrt[3]{x}! ${} \mathrel{\hat=} \sqrt[3]{x}$
    \end{itemize}

\end{frame}

\section{Das Paket \texttt{amsmath}}

\begin{frame}[fragile]
  \frametitle{\texttt{amsmath}}

  \onslide<+->

  \begin{block}{\textcolor{red!90}{Wichtig!}}
    Für das Setzen mathematischer Formeln sollte \emph{immer} das Paket \texttt{amsmath}
    geladen werden (oder ein Paket, welches \texttt{amsmath} lädt).
  \end{block}

  \onslide<+->

  Umgebungen aus dem Paket \texttt{amsmath}:

  \onslide<+->

  \begin{itemize}
  \item \texttt{equation} für einfache Formeln (ersetzt \texttt{displaymath})
  \item \texttt{split} für den Einsatz mehrzeiliger Formeln in der
    \texttt{equation}
  \item \texttt{align}, \texttt{alignat}, \texttt{aligned}, \texttt{alignedat}
    für mehrzeilige ausgerichtete Formeln
  \item \texttt{multline} für \enquote{zu lange} Formeln
  \item \texttt{gather} für lose zusammengeworfenen Formeln
  \item ...
  \end{itemize}
  \onslide<+->

  \texttt{*}-Variante für Umgebung ohne Nummer.

  \onslide<+->
  Empfehlenswert: Dokumentation von \texttt{amsmath} (\lstinline|texdoc amsmath|)

\end{frame}

% \section{Einzelelemente}

% % Grundregeln: Leerzeichen werden ignoriert, keine Leerzeilen, kein Text in Matheformeln

% \begin{frame}[fragile]
%   \frametitle{Klammern}

%   \onslide<+->

%   \LaTeX\ unterstützt jegliche Formen von Klammern:
%   \begin{equation*}
%     (x), \{x\},[x],\lfloor x\rfloor, \lceil x\rceil,\lvert x\rvert,\langle x \rangle,\ldots
%   \end{equation*}

%   \onslide<+->

%   \begin{lstlisting}
%     (x), \{x\},[x],\lfloor x\rfloor, \lceil x\rceil,
%     \lvert x\rvert,\langle x \rangle,\ldots
%   \end{lstlisting}
    
%   \medskip

%   \onslide<+->

%   Größenanpassung mit \lstinline{\left} und \lstinline{\right}, jeweils paarig:

% \begin{lstlisting}
% \left(\sum_{i=0}^1 5 = 10\right)
% \end{lstlisting}

%   \vspace*{-2ex}

%   \begin{equation*}
%     \left(\sum_{i=0}^1 5 = 10\right)
%   \end{equation*}

%   \vspace*{1ex}

%   \onslide<+->

% \begin{lstlisting}
% \left|\sum_{i=0}^1 5 = 10\right[
% \end{lstlisting}

%   \vspace*{-2ex}

%   \begin{equation*}
%     \left|\sum_{i=0}^1 5 = 10\right[
%   \end{equation*}

% \end{frame}

% \begin{frame}[fragile]
%   \frametitle{Klammern II}

%   \onslide<+->

%   oder auch

% \begin{lstlisting}
% \int_0^1 x^2 \mathop{} \mathsf d x =
%   \left.\frac{1}{3}x^3\right|_0^1
% \end{lstlisting}
%   \begin{equation*}
%     \int_0^1x^2 \mathop{} \mathsf d x = \left.\frac{1}{3}x^3\right|_0^1
%   \end{equation*}

%   denn \verb|.| ist das Sonderzeichen für eine leere Klammer.

%   \onslide<+->
%   \medskip

%   Klammern gibt es auch über- und unterhalb von Formeln mit Hilfe von
%   \lstinline{\underbrace} und \lstinline{\overbrace}:
% \begin{lstlisting}
% 1+\underbrace{ 2 + \overbrace{3 + 4}^{7} }_{9} = 10
% \end{lstlisting}
%   \begin{equation*}
%     1+\underbrace{ 2 + \overbrace{3 + 4}^{7} }_{9} = 10
%   \end{equation*}

% \end{frame}

% \begin{frame}[fragile]
%   \frametitle{Funktionen}

%   \onslide<+->

%   \begin{block}{Beobachtung}
%     \vspace*{-\baselineskip}
%     \begin{equation*}
%       sin(x) \neq \sin(x)
%     \end{equation*}
%   \end{block}

%   \onslide<+->

%   daher: allgemein gebräuchliche Funktionen werden gesondert behandelt:

%   \smallskip

%   \begin{center}
%     \begin{tabular}{ll|ll|ll|ll}
%       \toprule
%       \lstinline!\log! & $\log$ & \lstinline!\lg! & $\lg$ & \lstinline!\ln! & $\ln$ & \lstinline!\lim! & $\lim$ \\
%       \lstinline!\sin! & $\sin$ & \lstinline!\arcsin! & $\arcsin$ & \lstinline!\sinh! & $\sinh$ & \lstinline!\cos! & $\cos$ \\
%       \lstinline!\arccos! & $\arccos$ & \lstinline!\cosh! & $\cosh$ & \lstinline!\tan! & $\tan$ & \lstinline!\tanh! & $\tanh$ \\
%       \lstinline!\arctan! & $\arctan$ & \lstinline!\cot! & $\cot$ & \lstinline!\coth! & $\coth$ & \lstinline!\max! & $\max$ \\
%       \lstinline!\min! & $\min$  & \lstinline!\arg! & $\arg$ & \lstinline!\det! & $\det$  & \lstinline!\Pr! & $\Pr$ \\
%       \bottomrule
%     \end{tabular}
%   \end{center}

%   \smallskip

%   \onslide<+->

%   Damit sind auch korrekte Indizierungen möglich:
% \begin{lstlisting}
% lim_{n\to\infty} \frac{1}{n} = 0 \quad
%   \not\equiv \quad \lim_{n\to\infty}\frac{1}{n}=0
% \end{lstlisting}
%   \begin{equation*}
%     lim_{n\to\infty} \frac{1}{n} = 0 \quad \not\equiv \quad \lim_{n\to\infty}\frac{1}{n}=0
%   \end{equation*}
% \end{frame}

% \begin{frame}[fragile]
%   \frametitle{Mengen}

%   \onslide<+->

%   Mengenbedingungen werden mit dem Kommando \lstinline{\mid} gesetzt:
% \begin{lstlisting}
% \{\, n \in \mathbb N \mid n \ge \pi \,\}
% \end{lstlisting}
%   \vskip1ex
%   \begin{equation*}
%     \{\, n \in \mathbb N \mid n \ge \pi \,\}
%   \end{equation*}

%   \onslide<+->

%   Ein einfacher Strich | reicht nicht aus!
%   \begin{equation*}
%     \{\, n \in \mathbb N | n \ge \pi \,\}
%   \end{equation*}

% \end{frame}

% \begin{frame}[fragile]
%   \frametitle{Akzente}

%   \onslide<+->

%   Auch Akzente sind im Mathematikmodus neu:

%   \medskip

%   \begin{center}
%     \begin{tabular}{ll|ll|ll}
%       \toprule
%       \lstinline!\acute!    & $\acute x$    & \lstinline!\hat!       & $\hat x$          & \lstinline!\grave!   & $\grave x$ \\
%       \lstinline!\ddot!     & $\ddot x$     & \lstinline!\tilde!     & $\tilde x$        & \lstinline!\bar!     & $\bar x$ \\
%       \lstinline!\breve!    & $\breve x$    & \lstinline!\check!     & $\check x$        & \lstinline!\dot!     & $\dot x$ \\
%       \lstinline!\vec!      & $\vec x$      & \lstinline!\widetilde! & $\widetilde{xyz}$ & \lstinline!\widehat! & $\widehat{xyz}$ \\
%       \lstinline!\mathring! & $\mathring x$ & \lstinline!\prime!     & $x^{\prime}$        & \lstinline!'!        & $x'$\\
%       \bottomrule
%     \end{tabular}
%   \end{center}

%   \medskip

%   und mit Hilfe einiger Zusatzpakete gibt es noch viel mehr\ldots
% \end{frame}

% \begin{frame}[fragile]
%   \frametitle{\dots\ und noch viel mehr Symbole}

%   \onslide<+->

%   Hier:

%   \begin{center}
%     \small \url{http://tug.ctan.org/info/symbols/comprehensive/symbols-a4.pdf}
%   \end{center}

%   oder

% \begin{verbatim}
% $ texdoc symbols-a4.pdf
% \end{verbatim}

%   \onslide<+->

%   und noch viel mehr zu sagen, z.B.
%   \begin{itemize}
%   \item<+-> gibt es 8 unterschiedliche Symbolklassen
%   \item<+-> eigentlich 4 (8) Modi im Mathematikmodus
%   \item<+-> Text in mathematischen Formeln (mit \lstinline!\text{ein wenig Text}!)
%   \item<+-> manuelle Feinjustierung, z.B.
%     \begin{equation*}
%       \int_0^1x^2dx \neq \int_0^1 \! x^2 \mathop{} \mathrm{d}x
%     \end{equation*}
%   \item<+-> \ldots
%   \end{itemize}

% \end{frame}


% \section{Mathematische Aussagen}

% \begin{frame}
%   \frametitle{Definition, Satz, Beweis}

%   \onslide<+->

%   Für mathematische Argumentation sind oft spezielle Formatierungen für Definitionen,
%   Sätze, usw.\,nötig.

%   \onslide<+->

%   \begin{Lemma}
%     Ist $W$ eine ideale Welt, dann
%     \begin{enumerate}
%     \item werden alle Texte nur mit \LaTeX\ gesetzt, und deswegen
%     \item sind alle mathematischen Formeln schön!
%     \end{enumerate}
%   \end{Lemma}
%   \onslide<+->
%   \begin{Proof}
%     \onslide<+->Klar.\onslide<+->
%   \end{Proof}

% \end{frame}

% \begin{frame}[fragile]
%   \frametitle{Umgebungen für Defitionen, Sätze, \dots}

%   \onslide<+->

%   Manchmal werden solche Umgebungen von der Dokumentenklasse bereit gestellt (wie
%   z.B.\,\texttt{beamer}).

%   \onslide<+->

%   \medskip

%   Ansonsten helfen Pakete (\texttt{amsthm}, \texttt{ntheorem}, \dots)

%   \onslide<+->

% \begin{lstlisting}
% \usepackage[thmmarks,amsmath,hyperref]{ntheorem}
% \theoremstyle{standard}
% \theoremheaderfont{\normalfont\bfseries}
% \theorembodyfont{\slshape}
% \newtheorem{Theorem}     {Theorem} [section]
% \newtheorem{Proposition} [Theorem] {Proposition}
% \newtheorem{Lemma}       [Theorem] {Lemma}
% \newtheorem{Corollary}   [Theorem] {Corollary}
% \end{lstlisting}

% \end{frame}

% \section{Zum Abschluß}

% \begin{frame}
%   \frametitle{Eine wichtige Grundregel}

%   \onslide<2->

%   \begin{center}
%     \Huge\color[rgb]{0.9,0.1,0.1}

%     Lesbarkeit geht vor!
%   \end{center}

% \end{frame}

\end{document}

%%% Local Variables:
%%% mode: latex
%%% TeX-master: t
%%% TeX-engine: luatex
%%% ispell-local-dictionary: "de_DE"
%%% End:
