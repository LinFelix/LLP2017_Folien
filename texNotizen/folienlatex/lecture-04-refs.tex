\errorcontextlines=5

%%%

\documentclass[
%handout,
]{beamer}

\usepackage{ifluatex}
\ifluatex\else\errmessage{This document requires LuaLaTeX}\fi

\usepackage{etex,etoolbox}
\usepackage{fontspec}
\usepackage[ngerman]{babel}
\usepackage{csquotes}
\usepackage{array}
\usepackage{wrapfig}
\usepackage{booktabs}
\usepackage{ccicons}
\usepackage{calc}

\usepackage{tikz}
\usetikzlibrary{arrows,intersections,calc,through,%
  external,positioning,automata,datavisualization,%
  datavisualization.formats.functions}

\usepackage{luacode}
\usepackage{pgfplots}
\usepackage{manfnt}

%%% title and such

\title{Wissenschaftliches Arbeiten mit \LaTeX}
\author{\texorpdfstring{Felix Hilsky\\basierend auf einem Kurs von\\Daniel Borchmann,\\Tom Hanika und \\Max Marx}{Felix Hilsky basierend auf einem Kurs von Daniel Borchmann, Tom Hanika und Max Marx}}
\titlegraphic{\ccLogo \ccAttribution \ccShareAlike}

%%% theme

\usepackage{tikz}
\usetikzlibrary{shapes.multipart}
\usetheme{CambridgeUS}
\setbeamertemplate{blocks}[rounded][shadow=false]
\setbeamertemplate{items}{\raisebox{0.3ex}{%
    \tikz[scale=0.13] \draw[fill] (0,0) -- (0,1) -- (0.9,0.5) -- cycle;}}
\usetikzlibrary{arrows}
\tikzset{>={stealth'[sep]}}
\setbeamertemplate{navigation symbols}{}
\setbeamertemplate{footline}{}
\setlength{\abovedisplayskip}{0pt}
\setbeamerfont{title}{series=\bfseries}
\defbeamertemplate{block alerted begin}{bends}{%
  \begin{columns}
    \begin{column}{0.05\linewidth}
      \dbend
    \end{column}
    \begin{column}{0.95\linewidth}
      \vskip.75ex\usebeamercolor[fg]{block title
        alerted}\insertblocktitle{}
      \vskip.1em
      \usebeamercolor[fg]{normal text}
}
\defbeamertemplate{block alerted end}{bends}{%
    \end{column}
  \end{columns}
}
%%%

\mode<handout>{
  \usepackage{pgfpages}
  \pgfpagesuselayout{2 on 1}[a4paper,border shrink=5mm]
}

%%% lecture organization

\usepackage{xparse}
\DeclareDocumentCommand \Lecture { m m }{%
  \lecture{#1}{#2}
  \part{#1}
  \include{#2}
}

\AtBeginSection{
  \setbeamertemplate{blocks}[rounded][shadow=true]
  \begin{frame}[plain]
    \begin{block}{}
      \begin{center}
        \textcolor{darkred}{\textbf{\Large \strut\smash{\insertpart}}}\\[1ex]
        \textcolor{blue!70!black}{\strut\smash{\insertsection}}
      \end{center}
    \end{block}
  \end{frame}
  \setbeamertemplate{blocks}[rounded][shadow=false]
  \setbeamertemplate{block alerted begin}[bends]
  \setbeamertemplate{block alerted end}[bends]
}

%%% misc

\newcommand{\GNULinux}{GNU\lower-0.25ex\hbox{/}Linux}
\newcommand{\TikZ}{Ti\emph{k}Z}

\usepackage{listings}

\lstset{language=[LaTeX]TeX, basicstyle=\ttfamily,
  keywordstyle={\color{blue}\bfseries}, frame=tb, extendedchars=true, literate=%
  {ä}{{\"a}}1 {ö}{{\"o}}1, escapeinside={(*@}{@*)}, mathescape=true,
  basewidth=0.5em, keywordstyle={\color{blue}},
  morekeywords={[0]includegraphics,rotatebox,scalebox,resizebox,providecommand,
    subsection,subsubsection,paragraph,subparagraph,part,chapter,tableofcontents,
    mathring,text,mathbb,printindex,addbibresource,printbibliography,subtitle,
    institute,titlegraphic,subject,keywords,draw,path,color,textcolor,toprule,
    midrule,bottomrule,maketitle,setlength,enquote,listoffigures,listoftables,
    theoremstyle,theoremheaderfont,theorembodyfont,newblock,parencite,footcite,
    autocite,bibitem,middle,tikzset,usetikzlibrary,coordinate,node,foreach,
    datavisualization,varepsilon,autocite,bibitem,DeclareRobustCommand,
    DeclareDocumentCommand,IfBooleanTF,bye,frametitle,setbeamertemplate,pause,
    onslide,uncover,visible,invisible,only,alt,temporal,alert,AtBeginSection,
    usetheme,setbeamerfont,tikz,includeonlyframes,mode,pgfpagesuselayout,RequirePackage,
  },
}

\AtBeginDocument{\frame[plain]{\maketitle}}

%%% end of preamble
\subtitle{Nummerierung, Referenzierung, Bibliographie}
\date{2017-05-16}

\begin{document}

\section{Referenzieren}

\begin{frame}[fragile]
  \frametitle{Verweise im Dokument}

  \onslide<+->

  \LaTeX\ erlaubt die automatische Erstellung von Verweisen innerhalb des Dokuments

  \begin{itemize}
  \item<+-> mit dem Befehl \lstinline!\label{label-name}! wird ein \emph{Label} im Dokument
    gesetzt
  \item<+-> mit dem Befehl \lstinline!\ref{label-name}! wird auf dieses Label verwiesen
  \end{itemize}

  \onslide<+->

\begin{lstlisting}
\section{Einführung}
\label{sec:introduction}

Das Problem, welches wir behandeln wollen, ist wichtig!

\section{Das Problem}

Siehe Abschnitt~\ref{sec:introduction}!
\end{lstlisting}

  \onslide<+->

  \emph{Wichtig}: Zweimaliges Übersetzen notwendig! (Machen viele Editoren automatisch.)
  % beim ersten Mal werden die Informationen für die Label gesammelt. Beim zweiten Mal bei den Referenzierungen eingesetzt.
  % Demonstrieren: beim ersten Mal entstehen ??.
  % Demonstrieren: Zeilen in aux-Datei

\end{frame}

\begin{frame}[fragile]
  \frametitle{Platzierung von Labeln}

  \onslide<+->

  % Die Formatierung von \lstinline!\ref{label-name}! hängt von dem Verweis ab.
  % was wollte Daniel uns damit sagen?
  Faustregel: alles, was eine Nummer hat, kann mit einem Label versehen werden.

  \onslide<+->
% an Konvention kat:name erinnern
\begin{lstlisting}
\section{Abschnitt}
\label{sec:section}         % Verweis auf Abschnittsnummer
\begin{enumerate}
\item\label{item:eintrag} Eintrag % Verweis auf Einzelpunkt
\end{enumerate}
\begin{figure}
  $\dots$                 % Verweis auf Abbildung
  \caption{\label{fig:mein Bild} Bildunterschrift}
\end{figure}
\begin{equation}
  \int a dx = b \label{eq:int} % Verweis auf Formel
\end{equation} % bei align kann auf jede Zeile verwiesen werden
\end{lstlisting}

  \onslide<+->

  Verweis auf die Seitenzahl mit \lstinline!\pageref{label-name}!.

\end{frame}

\begin{frame}[fragile]
  \frametitle{Nützliche Pakete}

  \onslide<+->

  Es gibt einige nützliche Pakete, die Verweise besser formatieren können

  \begin{itemize}
  \item<+-> \lstinline!amsmath! gibt den Befehl \lstinline!\eqref{eq:ref}!, der die Klammern um den Verweis setzt: %\eqref{eq:int}.
  \item<+-> \lstinline!ntheorem! gibt den Befehl \lstinline!\thref{thm:main-theorem}!,
    welcher automatisch den Typ der Aussage hinzufügt (Satz~5.1, Lemma~5.1, Bemerkung~5.1,
    \dots)
  \item<+-> \lstinline!cleveref! gibt \lstinline!\cref! und weitere Befehle, welche
    automatisch den Typ der Referenz hinzufügen
  \item<+-> \lstinline!varioref! gibt \lstinline!\vref!, \lstinline!\vpageref!, und
    weitere, welche intelligente Formatierungen abhängig vom Abstand zwischen Referenz und
    Verweis erlauben
  \end{itemize}

\end{frame}

\end{document}