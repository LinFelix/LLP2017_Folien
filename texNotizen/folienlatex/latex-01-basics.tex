\errorcontextlines=5

%%%

\documentclass[
%handout,
]{beamer}

\usepackage{ifluatex}
\ifluatex\else\errmessage{This document requires LuaLaTeX}\fi

\usepackage{etex,etoolbox}
\usepackage{fontspec}
\usepackage[ngerman]{babel}
\usepackage{csquotes}
\usepackage{array}
\usepackage{wrapfig}
\usepackage{booktabs}
\usepackage{ccicons}
\usepackage{calc}

\usepackage{tikz}
\usetikzlibrary{arrows,intersections,calc,through,%
  external,positioning,automata,datavisualization,%
  datavisualization.formats.functions}

\usepackage{luacode}
\usepackage{pgfplots}
\usepackage{manfnt}

%%% title and such

\title{Wissenschaftliches Arbeiten mit \LaTeX}
\author{\texorpdfstring{Felix Hilsky\\basierend auf einem Kurs von\\Daniel Borchmann,\\Tom Hanika und \\Max Marx}{Felix Hilsky basierend auf einem Kurs von Daniel Borchmann, Tom Hanika und Max Marx}}
\titlegraphic{\ccLogo \ccAttribution \ccShareAlike}

%%% theme

\usepackage{tikz}
\usetikzlibrary{shapes.multipart}
\usetheme{CambridgeUS}
\setbeamertemplate{blocks}[rounded][shadow=false]
\setbeamertemplate{items}{\raisebox{0.3ex}{%
    \tikz[scale=0.13] \draw[fill] (0,0) -- (0,1) -- (0.9,0.5) -- cycle;}}
\usetikzlibrary{arrows}
\tikzset{>={stealth'[sep]}}
\setbeamertemplate{navigation symbols}{}
\setbeamertemplate{footline}{}
\setlength{\abovedisplayskip}{0pt}
\setbeamerfont{title}{series=\bfseries}
\defbeamertemplate{block alerted begin}{bends}{%
  \begin{columns}
    \begin{column}{0.05\linewidth}
      \dbend
    \end{column}
    \begin{column}{0.95\linewidth}
      \vskip.75ex\usebeamercolor[fg]{block title
        alerted}\insertblocktitle{}
      \vskip.1em
      \usebeamercolor[fg]{normal text}
}
\defbeamertemplate{block alerted end}{bends}{%
    \end{column}
  \end{columns}
}
%%%

\mode<handout>{
  \usepackage{pgfpages}
  \pgfpagesuselayout{2 on 1}[a4paper,border shrink=5mm]
}

%%% lecture organization

\usepackage{xparse}
\DeclareDocumentCommand \Lecture { m m }{%
  \lecture{#1}{#2}
  \part{#1}
  \include{#2}
}

\AtBeginSection{
  \setbeamertemplate{blocks}[rounded][shadow=true]
  \begin{frame}[plain]
    \begin{block}{}
      \begin{center}
        \textcolor{darkred}{\textbf{\Large \strut\smash{\insertpart}}}\\[1ex]
        \textcolor{blue!70!black}{\strut\smash{\insertsection}}
      \end{center}
    \end{block}
  \end{frame}
  \setbeamertemplate{blocks}[rounded][shadow=false]
  \setbeamertemplate{block alerted begin}[bends]
  \setbeamertemplate{block alerted end}[bends]
}

%%% misc

\newcommand{\GNULinux}{GNU\lower-0.25ex\hbox{/}Linux}
\newcommand{\TikZ}{Ti\emph{k}Z}

\usepackage{listings}

\lstset{language=[LaTeX]TeX, basicstyle=\ttfamily,
  keywordstyle={\color{blue}\bfseries}, frame=tb, extendedchars=true, literate=%
  {ä}{{\"a}}1 {ö}{{\"o}}1, escapeinside={(*@}{@*)}, mathescape=true,
  basewidth=0.5em, keywordstyle={\color{blue}},
  morekeywords={[0]includegraphics,rotatebox,scalebox,resizebox,providecommand,
    subsection,subsubsection,paragraph,subparagraph,part,chapter,tableofcontents,
    mathring,text,mathbb,printindex,addbibresource,printbibliography,subtitle,
    institute,titlegraphic,subject,keywords,draw,path,color,textcolor,toprule,
    midrule,bottomrule,maketitle,setlength,enquote,listoffigures,listoftables,
    theoremstyle,theoremheaderfont,theorembodyfont,newblock,parencite,footcite,
    autocite,bibitem,middle,tikzset,usetikzlibrary,coordinate,node,foreach,
    datavisualization,varepsilon,autocite,bibitem,DeclareRobustCommand,
    DeclareDocumentCommand,IfBooleanTF,bye,frametitle,setbeamertemplate,pause,
    onslide,uncover,visible,invisible,only,alt,temporal,alert,AtBeginSection,
    usetheme,setbeamerfont,tikz,includeonlyframes,mode,pgfpagesuselayout,RequirePackage,
  },
}

\AtBeginDocument{\frame[plain]{\maketitle}}

%%% end of preamble
\subtitle{Einführung}
\date{Sommersemester 2017}

\begin{document}
\section{Aufbau eines \LaTeX-Dokuments}

\begin{frame}[fragile]
  \frametitle{Standard-Dokumentenklassen}
  \begin{itemize}
    \item Spezifiziert das allgemeine Aussehen des Dokuments (Artikel, Report, Buch,
    Brief, \dots)
    \item Wird (im allgemeinen) als erstes im Dokument angegeben
  \end{itemize}  
  \begin{description}
    \item[article] Standardklasse zum Erstellen von einfachen Dokumenten
    \item[report] Standardklassen zum Erstellen längerer Arbeiten
    \item[book] Standardklassen zum Erstellen von Büchern
    \item[scrartcl, scrreprt, scrbook] KOMA-Script Varianten von article, report, book mit
      europäischen Standardwerten
    \item[memoir] Individuell anpassbare Dokumentenklasse
    \item[standalone] Minimale Seitnegröße für Graphiken
  \end{description}
  \begin{itemize}
    \item<+-> Können Optionen bekommen
    \begin{lstlisting}
      \documentclass[a4paper,english,draft]{article}
    \end{lstlisting}
  \end{itemize}
\end{frame}

\begin{frame}[fragile]
  \frametitle{Die Präambel}
  \onslide<1->

  \begin{itemize}
  \item<1-> Wird verwendet, um
    \begin{itemize}
    \item<2-> Pakete einzubinden
    \item<3-> Standardwerte des Dokuments anzupassen
    \item<4-> separate Befehle (\emph{Makros}) zu definieren oder zu ändern
    \end{itemize}
  \item<2-> Pakete werden eingebunden mittels
\begin{lstlisting}
\usepackage[(*@\textit{option}@*)]{(*@\textit{paketname}@*)}
\end{lstlisting}
  \end{itemize}

\end{frame}

\begin{frame}[fragile]
  \frametitle{Der \enquote{Dokumentenkörper}}

  Das eigentliche Dokument wird nun zwischen \lstinline!\begin{document}! und
    \lstinline!\end{document}! angegeben.  Dabei kann der Text \enquote{fast} beliebig
  eingegeben werden.  Dieser wird dann von \TeX\ ent- und ansprechend formatiert.

\begin{lstlisting}
\begin{document}
Bei Fülltexten sollte man drauf achten, dass es nicht
allzu viel Sinn macht, lange darüber nachzudenken, was
man wie schreibt.
\end{document}
\end{lstlisting}

  wird zu

  \begin{center}
    \parbox{0.8\linewidth}{\rm Bei Fülltexten sollte man drauf achten, dass es
      nicht allzu viel Sinn macht, lange darüber nachzudenken, was man wie schreibt.}
  \end{center}
\end{frame}

\section{Makros}

\begin{frame}[fragile]
  \frametitle{Was sind und was sollen Makros?}

  \onslide<+->

  Makros steuern die Formatierung des Textes in \LaTeX:

  \begin{itemize}
  \item<+-> sie beginnen mit \textbackslash, gefolgt von einer Folge von Buchstaben (keine
    Zahlen!)
  \item<+-> Mögliche Formen von Argumenten sind
    \begin{itemize}
    \item \lstinline!{$\textit{Argument}$}! bezeichnet \emph{obligatorische} Argumente
    \item \lstinline![$\textit{Argument}$]! bezeichnet \emph{optionale} Argumente
    \item \lstinline!($\textit{x},\textit{y}$)! oder \lstinline!<$\dots$>! sind auch
      üblich
    \end{itemize}

    Aber das ist \enquote{nur} Konvention!
  \end{itemize}
\end{frame}

\begin{frame}[fragile]
  \frametitle{Eigene Makros}
  \onslide<+->

  Eigene Makros ermöglichen (komplexe) Formatierungen und verringern Schreibaufwand.  In
  \LaTeX\ werden eigene Makros durch
\begin{lstlisting}
\newcommand{$\textit{Name}$}[$\textit{Anzahl\_der\_Argumente}$]{$\textit{Text}$}
\end{lstlisting}
  definiert. Dabei wird in Text jedes Vorkommen von \#\emph{Nummer} durch den Wert des
  \emph{Nummer}-ten Arguments \emph{syntaktisch} ersetzt.
\end{frame}

\begin{frame}[fragile]
  \frametitle{Weitere Makro-Definitions-Befehle}

  % \onslide<+->

  \begin{itemize}[<+->]
  \item Sollte es ein Makro schon geben, so redefiniert \lstinline!\renewcommand!
    dieses Makro
  \item \lstinline!\providecommand! definiert ein Makro nur, wenn es vorher noch nicht
    existierte
  \item Die Varianten \lstinline!\newcommand*!, \lstinline!\renewcommand*! bzw.\,
    \lstinline!\providecommand*! erlauben keine neuen Absätze in ihren Argumenten (und
    erlauben damit eine \enquote{bessere} Fehlerdiagnose, wenn man doch mal
    \texttt{\}} vergessen hat)
  \item Das Paket \texttt{xparse} bietet noch umfangreiche Möglichkeiten zur
    Makrodefinition, z.B.\ mehrere optionale Argumente.
  \end{itemize}
\end{frame}

\section{Quelltextformatierung: Nutze whitespace}

\begin{frame}[fragile]
  \frametitle{Quelltextformatierung}

  \onslide<+->

  Die Formatierung des Quelltextes ist (wie gesagt) \enquote{fast} beliebig.  Diese
  Formatierung wird allerdings nicht unbedingt im Dokument wieder gespiegelt.

  \begin{itemize}
  \item<+-> Zeilenumbrüche werden (fast) wie Leerzeichen interpretiert:
\begin{lstlisting}
Ich bin ein
Text.
\end{lstlisting}
    produziert den gleichen Code wie
\begin{lstlisting}
Ich bin ein Text.
\end{lstlisting}
  \item<+-> Doppelte Leerzeichen werden wie ein Leerzeichen interpretiert:
\begin{lstlisting}[showspaces=true]
Zwei  Leerzeichen
\end{lstlisting}
ist das gleiche wie
\begin{lstlisting}[showspaces=true]
Zwei Leerzeichen
\end{lstlisting}
  \end{itemize}

\end{frame}

\begin{frame}
  \frametitle{Absätze}
  Die Formatierung des Quelltextes ist (wie gesagt) \enquote{fast} beliebig.  Diese
  Formatierung wird allerdings nicht unbedingt im Dokument wieder gespiegelt.
  \begin{description}
    \item Leerzeilen und \textbackslash\lstinline!par! erzeugen einen neuen Absatz.
  \end{description}
\end{frame}

% \section{Textuelles Markup}

% \begin{frame}[fragile]
%   \frametitle{Abschnitte}

%   \begin{itemize}
%   \item<+-> geben die Grobstruktur des Dokuments an
%   \item<+-> In \LaTeX\ mit
%     \begin{itemize}
%     \item \lstinline!\section!, \lstinline!\section*!
%     \item \lstinline!\subsection!, \lstinline!\subsection*!
%     \item \lstinline!\subsubsection!, \lstinline!\subsubsection*!
%     \item \lstinline!\paragraph!, \lstinline!\paragraph*!
%     \item \lstinline!\subparagraph!, \lstinline!\subparagraph*!
%     \end{itemize}
%   \item<+-> *-Formen werden nicht nummeriert und treten auch nicht im
%     Inhaltsverzeichnis auf
%   \item<+-> Je nach Dokumentklasse sind auch noch möglich:
%     \begin{itemize}
%     \item \lstinline!\part!, \lstinline!\part*!
%     \item \lstinline!\chapter!, \lstinline!\chapter*!
%     \end{itemize}
%   \item<+-> Inhaltsverzeichnisse mit
% \begin{lstlisting}
% \tableofcontents
% \end{lstlisting}
%     und zweimaligem Übersetzen.
%   \end{itemize}

% \end{frame}

\end{document}
%%% Local Variables:
%%% mode: latex
%%% TeX-master: t
%%% TeX-engine: luatex
%%% ispell-local-dictionary: "de_DE"
%%% End:

%  LocalWords:  Textuelles Gliederungsstufe scrartcl scrreprt scrbook Script microtype
%  LocalWords:  Mikrotypographie geometry fontenc inputenc babel enumitem array booktabs
%  LocalWords:  listings hyperref amsmath amssymb mathtools ntheorem fragile
%  LocalWords:  Quelltextformatierung Seitenumbrüche Aufzählungstypen Formelelemente
