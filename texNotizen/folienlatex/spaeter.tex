% hier kommen Teile aus den Präsentationen hin, die man später nochmal erwähnenn sollte, aber an der Stelle, wo sie waren, zu früh sind.


% \begin{frame}[fragile]
%   \frametitle{Interpunktion}

%   \onslide<+->

%   Interpunktion (.\,,\,;\,- \dots) in \TeX\ funktioniert meist wie gewohnt, es gibt aber einige
%   Besonderheiten:

%   \begin{itemize}
%   \item<+-> Hinter .\ wird im allgemeinen ein größerer Abstand eingefügt:
% \begin{lstlisting}
% z. B. oder z.\,B.
% \end{lstlisting}
%     wird zu
%     \begin{center}
%       z. B. oder z.\,B.
%     \end{center}
%   \item<+-> \texttt{`\kern0pt`} und \texttt{'\kern0pt'} werden zu englischen
%     Anführungszeichen ``\dots''
%   \item<+-> \texttt{-\kern0pt-} wird zu --, ebenso \texttt{-\kern0pt-\kern0pt-} zu ---
%   \end{itemize}

% \end{frame}

% \begin{frame}[fragile,fragile]
%   \frametitle{Ausrichtung von Text}

%   \onslide<+->

% \begin{lstlisting}
% \begin{flushleft}
%   Dieser Text ist linksbündig.
% \end{flushleft}
% \end{lstlisting}

%   \onslide<+->

% \begin{lstlisting}
% \begin{flushright}
%   Dieser Text ist rechtsbündig.
% \end{flushright}
% \end{lstlisting}

%   \onslide<+->

% \begin{lstlisting}
% \begin{center}
%   Dieser Text ist zentriert
% \end{center}
% \end{lstlisting}

%   \onslide<+->

% \begin{lstlisting}
% \usepackage{ragged2e}
% \begin{justify}
%   Dieser Text ist im Blocksatz gesetzt.
% \end{justify}
% \end{lstlisting}

% \end{frame}

% \begin{frame}[fragile]
%   \frametitle{\textbf{Fett}, \textit{Kursiv} und Ähnliches}

%   \onslide<+->

%   Für das Markup einzelnen Wörter oder Sätze stehen die folgenden Kommandos zur Verfügung:
%   \bigskip

%   \centering
%   \begin{tabular}[c]{lcl}
%     \lstinline!\textbf{Text}! & $\leadsto$ & \textbf{Text}\\
%     \lstinline!\textsc{Text}! & $\leadsto$ & \textsc{Text}\\
%     \lstinline!\emph{Text}!   & $\leadsto$ & \emph{Text}\\
%     \lstinline!\textsf{Text}! & $\leadsto$ & \textsf{Text}\\
%     \lstinline!\textit{Text}! & $\leadsto$ & \textit{Text}\\
%     \lstinline!\textmd{Text}! & $\leadsto$ & \textmd{Text}\\
%     \lstinline!\textnormal{Text}! & $\leadsto$ & \textnormal{Text}\\
%     \lstinline!\textrm{Text}! & $\leadsto$ & \textrm{Text}\\
%     \lstinline!\textsl{Text}! & $\leadsto$ & \textsl{Text}\\
%     \lstinline!\texttt{Text}! & $\leadsto$ & \texttt{Text}\\
%     \lstinline!\textup{Text}! & $\leadsto$ & \textup{Text}
%   \end{tabular}

% \end{frame}

% \begin{frame}[fragile]
%   \frametitle{Schriftgröße}

%   \begin{onlyenv}<1>
%     \begin{block}{Logisch}
%       \centering
%       \begin{tabular}[c]{cc}
%         \lstinline!\tiny Text!         & \tiny Text \\
%         \lstinline!\scriptsize Text!   & \scriptsize Text \\
%         \lstinline!\footnotesize Text! & \footnotesize Text \\
%         \lstinline!\small Text!        & \small Text \\
%         \lstinline!\normalsize Text!   & \normalsize Text \\
%         \lstinline!\large Text!        & \large Text \\
%         \lstinline!\Large Text!        & \Large Text \\
%         \lstinline!\LARGE Text!        & \LARGE Text \\
%         \lstinline!\huge Text!         & \huge Text \\
%         \lstinline!\Huge Text!         & \Huge Text \\
%       \end{tabular}
%     \end{block}
%   \end{onlyenv}

%   \begin{onlyenv}<2>
%     \begin{block}{Manuell}

% \begin{lstlisting}
% \usepackage{graphicx}
% \scalebox{4}[3]{Beispieltext}
% \resizebox{5cm}{1cm}{Beispieltext 2}
% \end{lstlisting}

%       wird zu

%       \bigskip

%       \scalebox{4}[3]{Beispieltext 1} \resizebox{5cm}{1cm}{Beispieltext 2}

%     \end{block}
%   \end{onlyenv}

% \end{frame}
