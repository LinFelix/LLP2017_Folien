% Wir möchten am Ende dieser DS in der Lage sein unsere Hausaufgaben getext abzugeben.
% Als Beispiel nutze ich meine HANA HA von letzter Woche.

% wie letzte Woche gesehen:
% Dokumentenklasse gibt an, was für ein Dokument wir schreiben
% scrartcl ist die Variante des KOMA-Skriptes für Artikel, also kürzere Texte
% KOMA-Skript ist eine vielseitiges Paket für deutsche/ europäische Formate
\documentclass{scrartcl}

% ---------------
% genutzte Pakete
% ---------------

% das Ü wird nicht angezeigt
% das liegt daran, dass LaTeX standardmäßig nicht das Speicherformat UTF-8 erkennt. Das muss man ihm mitteilen:
% damit sollte (fast) alle westeuropäischen Buchstaben erkannt werden
\usepackage[utf8]{inputenc}

% das Ü wird jetzt als U plus Punkten eingefügt
% unter anderem kann es damit in manchen pdf-Programmen nicht gesucht werden
% mit fontenc kann eingestellt werden, tatsächlich mehr Zeichen zu nutzen
% Standard für westeuropäische Texte:
\usepackage[T1]{fontenc}

% beim Kompilieren kommt der Fehler "Undefined control sequence" für \text
% Lösung: nutze Paket amsmath
% amsmath sollte immer benutzt werden, wenn man den Mathematikmodus nutzt
\usepackage{amsmath}

% --------------
% eigene Befehle
% --------------

% Absolutbetrag
% nutze \lvert und \rvert statt | um die Abstände von linken und rechten Trennsymbolen ("Delimiter") zu nutzen
\newcommand{\abs}[1]{\lvert #1 \rvert}

% Norm
% analog mit doppelten Strichen
\newcommand{\norm}[1]{\lVert #1 \rVert}

% Es gibt (alle) weiteren Trennsymbole
% siehe Table 214, 215, 216 in texdoc symbols
% z.B. \langle, \rangle für ⟨, ⟩
% ], [
% \rceil (⌉), \lceil, \lfloor, \rfloor für Auf- und Abrunden
% \{, \} für geschwungene Klammern

% ----------
% Basic info
% ----------

% Gib Autor, Titel und Datum an
% wird bei \maketitle genutzt
\author{Felix Hilsky, Matr.nr. 4122148}
\title{HANA PDE Übungsblatt 2}
\date{2017-05-09}

% zwischen \begin{document} und \end{document} steht der Text
\begin{document}
  % \maketitle erstellt eine einfache Überschrift
  % meißt wird man den Titel selbst gestalten
  % aber jetzt interessiert uns erstmal der Inhalt
  \maketitle

  % 1. Aufgabe verkürzt hier wiedergegeben

  Gegeben sei das Gebiet
  % Omega ist eine Variable und damit im mathematischen Modus
  % dies ist "inline-Modus", also Mathematik im fließenden Text
  $\Omega$ mit
  % vergleiche "-" innerhalb und außerhalb von \text
  % inhaltlich ist das "-" kein Minus, sondern ein Bindestrich, also Text
  $C^1\text{-Rand}$ und das auf hinreichend glatten Funktionen definierte Funktional $I$
  % inline-Math kann auch mit \(Math\) oder \begin{math} Math \end{math} angeschaltet werden. Alle drei Varianten tun das gleiche, aber es ist empfehlenswert, konsistent eine zu verwenden
  %
  % equation ist eine Möglichkeit für "abgesetzte Formeln"
  % etwas größere Formeln sollten abgesetzt werden
  % equation kommt auch aus dem Paket amsmath
  \begin{equation}
    % \int ist das Integral-Zeichen
    I(u) = \int_{\Omega}
    % \sqrt ist das Wurzelzeichen. Im Argument steht, woraus die Wurzel gezogen wird
    \sqrt
    % Betrag mit dem Befehl \abs, selbstdefiniert!
    { \abs{ \nabla u (x) } ^ 2 + 1} dx.
  \end{equation}

\end{document}
