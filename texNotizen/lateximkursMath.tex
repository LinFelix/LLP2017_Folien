% Wir möchten am Ende dieser DS in der Lage sein unsere Hausaufgaben getext abzugeben.
% Als Beispiel nutze ich meine HANA HA von letzter Woche.

% wie letzte Woche gesehen:
% Dokumentenklasse gibt an, was für ein Dokument wir schreiben
% scrartcl ist die Variante des KOMA-Skriptes für Artikel, also kürzere Texte
% KOMA-Skript ist eine vielseitiges Paket für deutsche/ europäische Formate
\documentclass{scrartcl}

% ---------------
% genutzte Pakete
% ---------------

% Standardpakete, die immer geladen werden sollten

% das Ü wird nicht angezeigt
% das liegt daran, dass LaTeX standardmäßig nicht das Speicherformat UTF-8 erkennt. Das muss man ihm mitteilen:
% damit sollte (fast) alle westeuropäischen Buchstaben erkannt werden
\usepackage[utf8]{inputenc}

% das Ü wird jetzt als U plus Punkten eingefügt
% unter anderem kann es damit in manchen pdf-Programmen nicht gesucht werden
% mit fontenc kann eingestellt werden, tatsächlich mehr Zeichen zu nutzen
% Standard für westeuropäische Texte:
\usepackage[T1]{fontenc}
% Die Umgebung "proof" startet mit dem Wort "proof". Das passt nicht in einen deutschen Text
% automatisch generierte Texte werden mit dem Paket babel in die angegebene Sprache übersetzt
% babel sorgt auch für deutsche Silbentrennung und weitere Feinheiten
% deshalb ist babel ein Standardpaket, was immer geladen werden sollte (auch für englische Texte)
% ngerman = new german (neue deutsche Rechtschreibung)
\usepackage[ngerman]{babel}

% Standardpakete für Mathematik, die man immer braucht

% beim Kompilieren kommt der Fehler "Undefined control sequence" für \text
% Lösung: nutze Paket amsmath
% amsmath sollte immer benutzt werden, wenn man den Mathematikmodus nutzt
\usepackage{amsmath}
% amssymb liefert diverse Symbole für den Mathemodus
% unter anderem \mathbb für doppelt gestrichene Buchstaben (bb = blackboard)
\usepackage{amssymb}
% amsthm bietet Formatierung für Sätze, Bemerkungen, Lemmas, Aufgaben, ...
\usepackage{amsthm}
% für mehr Einstellungsmöglichkeiten gibt es zusätzlich \usepackage{ntheorem}

% --------------
% eigene Befehle
% --------------

% Absolutbetrag
% nutze \lvert und \rvert statt | um die Abstände von linken und rechten Trennsymbolen ("Delimiter") zu nutzen
\newcommand{\abs}[1]{\left\lvert #1 \right\rvert}

% Norm
% analog mit doppelten Strichen
\newcommand{\norm}[1]{\left\lVert #1 \right\rVert}

% Es gibt (alle) weiteren Trennsymbole
% siehe Table 214, 215, 216 in texdoc symbols
% z.B. \langle, \rangle für ⟨, ⟩
% ], [
% \rceil (⌉), \lceil, \lfloor, \rfloor für Auf- und Abrunden
% \{, \} für geschwungene Klammern

% \left und \right sorgen dafür, dass Delimiter sich an die Größe anpassen
% sie müssen immer in Paaren auftreten

% Rand einer Menge
% \partial sollte nicht im Text auftauchen, da "partial" nichts mit dem Rand, sondern partiellen Ableitungen zu tun hat und so zu Verwirrung führt
\newcommand{\Rand}[1]{\partial #1}

% Reelle Zahlen
% \mathbb macht das Doppeltgestriche R
\newcommand{\reals}{\mathbb{R}}

% Natürliche Zahlen
\newcommand{\naturals}{\mathbb{N}}

% Ganze Zahlen
\newcommand{\integers}{\mathbb{Z}}

% Rationale Zahlen
\newcommand{\rationals}{\mathbb{Q}}

% Abschluss einer Menge
\newcommand{\closure}[1]{\overline{#1}}

% --------------------------------
% eigene Umgebungen (environments)
% --------------------------------

% bietet die Umgebung "problem" mit dem Titel "Aufgabe"
% Formatierung ist die, die die AMS für Theoreme vorsieht
\newtheorem {problem} {Aufgabe}

% alle weiteren environments, die über \newtheorem definiert werden,
% bekommen den Stil "remark"
% die möglichen Stile sind plain (Standard), definition und remark
% Erläuterung siehe "texdoc amsthm" "4.1 The \theoremstyle command"
\theoremstyle{remark}

% Bemerkungen
\newtheorem {remark} {Bemerkung}

% newenvironment bietet eine neue Umgebung (unabhängig von amsthm)
% 1. Arg: Name der Umgebung
% 2. Arg: Code, der am Anfang geschrieben wird (bei \begin{loesung})
% 3. Arg: Code, der am Ende geschrieben wird (bei \end{loesung})
\newenvironment{loesung}
  % Das optionale Argument von Proof enthält den Ersatz vom Wort Proof am Anfang des Beweises
  {\begin{proof}[Lösung]}
  {\end{proof}}

% ----------
% Basic info
% ----------

% Gib Autor, Titel und Datum an
% wird bei \maketitle genutzt
\author{Felix Hilsky, Matr.nr.\ 4122148}
\title{HANA PDE Übungsblatt 2}
\date{2017-05-09}

% zwischen \begin{document} und \end{document} steht der Text
\begin{document}
  % \maketitle erstellt eine einfache Überschrift
  % meißt wird man den Titel selbst gestalten
  % aber jetzt interessiert uns erstmal der Inhalt
  \maketitle

  % Umgebung für eine Aufgabe (im Englischen heißt das immer problem)
  % Nummerierung passiert automatisch
  % Optional kann man einen Titel angeben
  \begin{problem}[Variationsprinzip und Plateau-Problem]
    
  % 1. Aufgabe verkürzt hier wiedergegeben

  Gegeben sei das Gebiet
  % Omega ist eine Variable und damit im mathematischen Modus
  % dies ist "inline-Modus", also Mathematik im fließenden Text
  $\Omega$ mit
  % vergleiche "-" innerhalb und außerhalb von \text
  % inhaltlich ist das "-" kein Minus, sondern ein Bindestrich, also Text
  $C^1\text{-Rand}$ und das auf hinreichend glatten Funktionen definierte Funktional $I$
  % inline-Math kann auch mit \(Math\) oder \begin{math} Math \end{math} angeschaltet werden. Alle drei Varianten tun das gleiche, aber es ist empfehlenswert, konsistent eine zu verwenden
  %
  % equation ist eine Möglichkeit für "abgesetzte Formeln"
  % etwas größere Formeln sollten abgesetzt werden
  % equation kommt auch aus dem Paket amsmath
  \begin{equation}
    % \int ist das Integral-Zeichen
    I(u) = \int_{\Omega}
    % \sqrt ist das Wurzelzeichen. Im Argument steht, woraus die Wurzel gezogen wird
    \sqrt
    % Betrag mit dem Befehl \abs, selbstdefiniert!
    { \abs{ \nabla u (x) } ^ 2 + 1}
    % das d in Integralen ist etwas komplizierter
    % deshalb schreiben wir dafür einen Befehl
    d x.
  \end{equation}
  %
  % Für die Doppelpunkte, die in Funktionsdef. genutzt werden
  % nutze immer \colon, da : als binäre Relation falsche Abstände hat
  Sei $u_0 \colon \Rand{\Omega} \to
  % das doppelt gestrichene R hat einen langen Befehl, den man lieber auslagert für lesbaren Code und Änderbarkeit
  \reals$ stetig.
  % "." bedeutet für LaTeX, dass ein Satz beendet wird. Danach wird eine etwas größere Lücke gelassen. Bei Abkürzungen wie d.h. ist das falsch, daher sagen wir LaTeX, dass es ein normales Leerzeichen ist: \ 
  Sei $u$ ein Minimierer von $I$, d.h.\ 
  % Einschränkungen von Funktionen haben verschiedene Notationen
  % um mich jetzt nicht einzuschränken, nutze Makro
  $u = u_0$ auf $\Rand {\Omega}$ mit
  \begin{equation}
    % \leq ist "kleiner als"
    I(u) \leq I(v)
    % qquad macht einen Abstand
    \qquad
    % \forall ist das "für alle" Symbol
    % es liest sich meißt besser mit Text
    \text{für alle } v
    % \in ist das "ist Element von" Symbol
    \in
    % overline sagt nichts über die Bedeutung des Symbols aus, deshalb auslagern
    C^2(\closure{\Omega})
    % wie wir schon früher gesehen haben ist ^ für hochgestelltes
    % und _ für tiefgestelltes ("Super-, Subskripte")
    % 
    % Leerzeilen markieren Absätze, dürfen in Mathe-Umgebungen nicht auftauchen
    % deshalb Leerzeilen auskommentieren
    %
    % Text im Mathemodus wird erstmal als Variablen aufgefasst
    % daher müssen wir erwähnen, dass es sich um Text handelt
    \text{ mit } v = u_0 \text{ auf } \Rand {\Omega}.
    % was ich gerne vergesse: wenn ein Satz mit einer Formel aufhört, gehört da auch ein Punkt hin.
  \end{equation}
  Zeige, dass $u$ die folgende Minimalflächen-Gleichung löst:
  \begin{equation}
    % +, -, / sind einfach die Symbole
    - \nabla
    % \cdot für den Mal-Punkt oder hier den Skalarprodukt-Punkt
    \cdot
    % Brüche mit \frac{Zähler}{Nenner}
    \frac
      {\nabla u(x)}
      {\sqrt{\abs{\nabla u(x)}^2 + 1}}
    = 0 \text{ in } \Omega, \qquad u = u_0 \text{ auf } \Rand{\Omega}
  \end{equation}

  \end{problem}

  % Meine Lösung:
  \begin{loesung}
    Seien $\Omega$, $u \in C^2(\closure{\Omega})$, $I$ wie in der Aufgabe
    % \infty ist das Unendlich-Symbol
    % man kann Sub- und Superskripte kombinieren. Die Reihenfolge ist egal.
    und $\varphi \in C_C^{\infty}(\Omega)$ eine Richtung.
    Dann ist für $t \in \reals$ die Funktion $u + t\varphi$ im Definitionsbereich von $I$ und es muss
    % align ist ähnlich wie equation eine abgesetzte Mathematikumgebung
    % sie bietet aber die Möglichkeit mehrere Zeilen zu nutzen und aneinander auszurichten
    \begin{align}
      % der zusätzliche Platz vor dem d ist jetzt hinderlich
      % also neues Makro
      \left. \frac {d}{dt} I(u + t\varphi) \right|_{t = 0} & = 0 \\
      % Neuzeilen mit \\
      % Alle & werden untereinander gesetzt
      % typischerweise schreibt man sie vor Relationen wie =
      \frac {d}{dt} I(u + t\varphi)  &=
      \frac {d}{dt} \int_{\Omega} \sqrt{
        \abs{
          \nabla(u+tv) (x)
        }^2 + 1
      } d x\\
      & = \dots & \text{Magie}\\
      % Implikationen und Äquivalenzen sind passend benannt:
      % \implied, \iff, \impliedby
      % alternativ auch \because und \therefore
      \implies \left. \frac {d}{dt} I(u + t\varphi) \right|_{t = 0} & =
      \int_{\Omega} \frac
        {\nabla v (x) \cdot \nabla u(x)}
        {\sqrt{
          \abs{
            \nabla (u(x))
            }^2 + 1
          }
        }
      d x
      % \dots schreibt drei Punkte
      % es gibt auch \ldots, \vdots, \ddots, ...
      % siehe auch texdoc symbols Table 262 AMS dots (page 108)
    \end{align}

    % um die Ausrichtung innerhalb von align zu verstehen, wisse:
    % es handelt sich um eine Tabelle, deren Spalten durch & getrennt sind
    % und abwechselnd rechts- und linksbündig ausgerichtet sind
  \end{loesung}

  % Ein paar Details möchte ich noch zeigen. Die passen thematisch in eine Bemerkung, die die Notation erläutert.
  % Eine Bemerkung passt auch gut in eine Umgebung.

  \begin{remark}[Gradient]
    Für offenes $\Omega \subseteq \reals^d$ sei für eine Funktion $u \colon \Omega \to \reals$ der Gradient $\nabla u$ definiert als
    \begin{equation}
      \nabla u (x) =
      \begin{pmatrix} % Matrix mit runden Klammern, aus Paket amsmath
      % weitere Matrizen und Bemerkungen in "texdoc amsmath" "4.1 Matrices"
        \partial_1 u(x) \\
        \vdots \\
        \partial_d u(x)
      \end{pmatrix}.
      % wie bei align sind mehrere Spalten durch & zu trennen
    \end{equation}
    Der einfache Punkt $\cdot$ bezeichnet das Standardskalarprodukt, also
    \begin{equation}
      % Summen mit \sum
      % Sub und Superskripte werden darüber und darunter, nicht daneben gesetzt
      % analog gibt es \prod für Π
      \nabla v \cdot \nabla u = \sum_{i=1}^{d} \partial_i v \partial_i u
    \end{equation}
    Im aktuellen Aufgabenblatt geht es auch um gleichmäßige Konvergenz. Das wird auch mit dem üblichen $\lim$ Symbol geschrieben: $\lim_{h \to 0} u_h = u_0$ bzw.\
    \begin{equation}
    % lim, limsup, sin, cos, tan, viele weitere Funktionen
    % sind vordefiniert und sehen "richtig" aus als Makro
      \lim_{h \to 0} u_h = u_0 \text{ glm.} \iff u_h \rightrightarrows u_0
    % eigene solche Makros erstellt man mit \operatorname oder \DeclareMathOperator
    % 
    \end{equation}
  \end{remark}
\end{document}
