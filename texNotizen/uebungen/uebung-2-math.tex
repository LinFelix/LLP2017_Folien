\documentclass{latex-htw-uebung}

\usepackage{amssymb}
\usepackage{amsmath}
\title{2. Übungsblatt}
\date{2017-05-02}

\newcommand{\naturals}{\mathbb{N}}
\newcommand{\integers}{\mathbb{Z}}
\newcommand{\rationals}{\mathbb{Q}}
\newcommand{\intersec}{\cap}
\newcommand{\union}{\cup}
\newcommand{\logand}{\wedge}
\newcommand{\logor}{\vee}
\newcommand{\reals}{\mathbb{R}}
\newcommand{\complex}{\mathbb{C}}

\begin{document}

Schreibe ein paar Formeln. Schreibe deine Hausaufgaben oder die untenstehenden Formeln ab. 
Stößt du auf Symbole, deren \LaTeX-Makroname nichts mit der Bedeutung zu tun haben oder dir zu lang sind? Definiere Makros, die den Code lesbarer machen!
\begin{align}
  a + \frac {b}{3} = \gamma \\
  7^{\sum_{a = 1}^{\sigma{5}} \zeta(a)} \equiv \frac 35 \mod 2 \\
  \sum_{a = 1}^{\sigma(5)} \zeta(a) = \prod_{x \in \{y \in D : \beta(y) \notin \aleph\}} x^{-1} \cdot \vec {y} \times \vec {z}
\end{align}
Eine Formel im Text sollte nicht zu lang sein. Dies ist also eine schlechte Idee: $
  \naturals = \integers \cap \rationals_{+}  = \{z \in \complex \mid \Im (z) = 0 \logand
    \exists a, b \in \integers: z = \frac ab \logand
    ((a > 0 \logand b > 0) \logor (a < 0 \logand b < 0)\}$
Ändert sich das Aussehen dieser Gleichung im abgesetzten Mathemodus?
\end{document}

%%% Local Variables:
%%% mode: latex
%%% TeX-master: t
%%% ispell-local-dictionary: "de_DE"
%%% End:
