\documentclass{latex-htw-uebung}

\title{7. Übungsblatt}
\date{29.~Mai~2017}

\begin{document}

\NewTask Hier eine kleine BEAMER-Präsentation zum Anfangen.

\begin{lstlisting}
  \documentclass[ngerman]{beamer}
  \usepackage[utf8]{inputenc}\usepackage[T1]{fontenc}\usepackage{babel}

  \usetheme{Dresden}
  \title[Zeitreisen heute.]{Über die Möglichkeit von Zeitreisen}
  \author[Doc Brown et al.]{Doc Brown, Danny McFly, Tom McFly}
  \date{\today}
  \institute[TU Dresden]{Technische Universität Dresden}

  \begin{document}
  \frame{\titlepage}
  \section{Einleitung}
  \begin{frame}
    \frametitle{Historischer Überblick.}
    Wie wir seit H.\,G.\,Wells wissen \ldots
    \begin{block}{Zusammenfassung}<2->
      Der Flux-Kompensator ist wichtig!
    \end{block}
  \end{frame}

\end{document}
\end{lstlisting}
Nachdem du den Code eingegeben hast, führe die folgenden Schritte aus:
\begin{enumerate}
\item Ändere das genutzte BEAMER-Theme (\lstinline|usetheme|)
  nacheinander in \texttt{Copenhagen}, \texttt{Darmstadt},
  \texttt{Goettingen}, \texttt{PaloAlto}, \texttt{CambridgeUS}. Hast du einen Favoriten?
\item Ändere alle Daten nach eigenen Ideen, d.\,h.\ Autor, Titel,
  Hochschule, etc.
\item Füge ein weiteres \lstinline|frame| hinzu, welches einen von dir
  gewählten Titel trägt. Dieses \lstinline|frame| soll aus drei
  Blöcken bestehen, mit jeweils einer von dir gewählten
  Blocküberschrift sowie ein paar Wörtern Inhalt.
\item Verändere das eben erstellte \lstinline|frame|, so dass die
  Blöcke nacheinander erscheinen.
\item Positioniere einen der Blöcke neben den anderen beiden. (Hinweis: mehrspaltig)
\item Füge ein weiteres \lstinline|frame| hinzu, welches den Titel
  \enquote{Achtung} trägt, und eine Liste von fünf Stichpunkten, welche
  nacheinander aufgedeckt werden.

\item Füge folgendes \lstinline|frame| hinzu:
  \begin{lstlisting}
\begin{frame}\frametitle{Cover Story}
  Der folgende Text wird \uncover<2->{erst nach und
   nach} \only<3-4>{nicht} \uncover<1-4>{sinnvoll} \visible<6->{lesbar.}
\end{frame}
   \end{lstlisting}
Versuche anhand des entstandenen \lstinline|frame| das
unterschiedliche Verhalten von \lstinline|uncover|, \lstinline|visible|,
und \lstinline|only| zu verstehen. Schaue in der \lstinline|beamer|-Dokumentation nach, ob deine Beobachtungen das ganze Verhalten beschreibt.

% \item Lade das Paket \texttt{listings}  und füge folgendes \lstinline|frame| hinzu:
%   \begin{lstlisting}
% \begin{frame}[fragile]
%   \begin{lstlisting}[language=C]
%     #include <stdio.h>
%     main(){
%       printf("Hello World");
%     }
%   \end{(*@{lstlisting}{@*)}
% \end{frame}
%     \end{lstlisting}
% Man kann also leicht Quellcode darstellen.
% \item Lade das Paket \texttt{tikz} und füge das folgende
%   \lstinline|frame| hinzu:
%   \begin{lstlisting}
% \begin{frame}\frametitle{Spaß mit TikZ}
%   \begin{tikzpicture}[overlay,anchor=south west]
%     \draw[blue](0,0) circle(2);
%     \draw[red](7,3) circle(1);
%   \end{tikzpicture}
%   Wichtiger Text
% \end{frame}
%   \end{lstlisting}
% Du siehst, du kannst in den Frames \enquote{herummalen} mit TikZ. Male
% weiter mit deinen TiKZ-Kenntnissen in diesem Frame herum.

\item Entferne die (nervigen) Navigationssymbole (siehe Vortrag).

\end{enumerate}

\end{document}

%%% Local Variables:
%%% mode: latex
%%% TeX-master: t
%%% ispell-local-dictionary: "de_DE"
%%% End:
