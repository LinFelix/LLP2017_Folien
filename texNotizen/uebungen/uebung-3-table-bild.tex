\documentclass{latex-htw-uebung}

\usepackage{amssymb}
\usepackage{amsmath}
\title{3.\ Übungsblatt}
\date{2017-05-16}

\begin{document}

Schreibe das Dokument, das auf der Seite \url{myfsr.de/llp} als Bild verlinkt ist. Dabei muss das Resultat nicht genauso aussehen, sondern ein schönes Dokument werden. Um dir Tipparbeit zu ersparen, findest du im \url{github}-Repository eine Datei mit einem Gutteil der Texte (\url{uebung3texte.txt}).

Wenn du stattdessen einen eigenen Text texst, lasse es eine Tabelle und ein Bild beinhalten.

Für ein paar Teile können dir diese Hinweise hoffentlich helfen:
\begin{itemize}
  \item Modifikationen von Symbolen wie Punkte, Pfeile, Hüte, etc.\ finden sich als \enquote{accents} in der Symbolliste. (\url{ctan.org/pkg/comprehensive} oder \lstinline!texdoc symbols!)
  \item Mehrere Zeilen umspannende Klammern können mithilfe der \lstinline!aligned!-Umgebung des Paketes \lstinline!amsmath! erstellt werden. Schaue dir dafür die Dokumentation (Abschnitt \enquote{Alignment building blocks}) des Paketes an!
  \item Neben \lstinline!\mathbb! gibt es noch mehr besondere Arten Variablen im Mathematikmodus zu setzen. Auch diese finden sich in der Symbolliste.
  \item Das Paket \lstinline!hyperref! bietet (neben vielem anderen) das Makro \lstinline!\url{Link}!, dass \enquote{Link} als Link formatiert.
\end{itemize}
% Schreibe ein paar Formeln. Schreibe deine Hausaufgaben oder die untenstehenden Formeln ab. 
% Stößt du auf Symbole, deren \LaTeX-Makroname nichts mit der Bedeutung zu tun haben oder dir zu lang sind? Definiere Makros, die den Code lesbarer machen!
% \begin{align}
%   a + \frac {b}{3} = \gamma \\
%   7^{\sum_{a = 1}^{\sigma{5}} \zeta(a)} \equiv \frac 35 \mod 2 \\
%   \sum_{a = 1}^{\sigma(5)} \zeta(a) = \prod_{x \in \{y \in D : \beta(y) \notin \aleph\}} x^{-1} \cdot \vec {y} \times \vec {z}
% \end{align}
% Eine Formel im Text sollte nicht zu lang sein. Dies ist also eine schlechte Idee: $
%   \naturals = \integers \cap \rationals_{+}  = \{z \in \complex \mid \Im (z) = 0 \logand
%     \exists a, b \in \integers: z = \frac ab \logand
%     ((a > 0 \logand b > 0) \logor (a < 0 \logand b < 0)\}$
% Ändert sich das Aussehen dieser Gleichung im abgesetzten Mathemodus?
\end{document}

%%% Local Variables:
%%% mode: latex
%%% TeX-master: t
%%% ispell-local-dictionary: "de_DE"
%%% End:
