% Dokumentenklasse
% immer am Anfang des Dokuments
\documentclass{scrartcl}

% Präambel
\usepackage[utf8]{inputenc} % Eingabekodierung
\usepackage[T1]{fontenc}    % Schrifkodierung
\usepackage[ngerman]{babel} % optimiert für Deutsch (z.B. Titel, Silbentrennung)

\usepackage{amsmath}        % (nahezu) unerlässliches Paket für den Mathemodus
\usepackage{amssymb}        % ebenso

\usepackage{amsthm}       % Theorem-Umgebungen
\usepackage{hyperref}     % bietet viel für Verweise

% Erstellt neues Theorem-Umgebung mit Namen Theorem, das im Dokument Satz heißt und innerhalb von section nummeriert wird
\newtheorem{theorem}     {Satz} [section]
% Propositionen werden mit Theorem zusammen nummeriert
\newtheorem{proposition} [theorem] {Proposition}
\newtheorem{lemma}       [theorem] {Lemma}
\newtheorem{corollary}   [theorem] {Korollar}

\title{Mathematikmodus}
\author{LLP-Kurs}
\date{nach Ostern}

\newcommand{\setDef}[2]{\left\{#1 
  \,\middle\vert\, #2\right\}}

% Ende Präambel
% Beginn eigentliches Dokument
\begin{document}
\maketitle
  Nahezu alle mathematischen Paper nutzen als Strukturierung Sätze, Definitionen, Lemmata, etc. Das Paket amsthm bietet dafür Umgebungen.
  %
  \begin{theorem}
    Sei $A$ eine Menge. Dann ist $B \subset A \times A$ definiert durch
    \begin{equation*}
      B = \{(a,a) \in A \times A \mid a \in A\}
    \end{equation*}
    eine \emph{Äquivalenzrelation} auf $A$.
  \end{theorem}

  Um den obigen Satz zu beweisen, ist etwas Vorarbeit vonnöten. Dafür betrachte die folgende Proposition:
  \begin{proposition}
    Sei $a$ etwas. Dann ist $(a,a) = (a,a)$ und $a = a$.
  \end{proposition}
  \begin{proof}
    Dies ist klar aufgrund der Definition von $=$.
  \end{proof}

  Jetzt können wir den ersten Satz beweisen.
  \begin{proof}
    Also jetzt gehen mir die Ideen aus. Das ist halt wahr. Heureka!
  \end{proof}
\end{document}
