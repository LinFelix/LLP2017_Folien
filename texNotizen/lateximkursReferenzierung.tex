\documentclass{scrartcl}

% Standardpakete, die immer geladen werden sollten

% das Ü wird nicht angezeigt
% das liegt daran, dass LaTeX standardmäßig nicht das Speicherformat UTF-8 erkennt. Das muss man ihm mitteilen:
% damit sollte (fast) alle westeuropäischen Buchstaben erkannt werden
\usepackage[utf8]{inputenc}

% das Ü wird jetzt als U plus Punkten eingefügt
% unter anderem kann es damit in manchen pdf-Programmen nicht gesucht werden
% mit fontenc kann eingestellt werden, tatsächlich mehr Zeichen zu nutzen
% Standard für westeuropäische Texte:
\usepackage[T1]{fontenc}
% Die Umgebung "proof" startet mit dem Wort "proof". Das passt nicht in einen deutschen Text
% automatisch generierte Texte werden mit dem Paket babel in die angegebene Sprache übersetzt
% babel sorgt auch für deutsche Silbentrennung und weitere Feinheiten
% deshalb ist babel ein Standardpaket, was immer geladen werden sollte (auch für englische Texte)
% ngerman = new german (neue deutsche Rechtschreibung)
\usepackage[ngerman]{babel}

% Standardpakete für Mathematik, die man immer braucht

% beim Kompilieren kommt der Fehler "Undefined control sequence" für \text
% Lösung: nutze Paket amsmath
% amsmath sollte immer benutzt werden, wenn man den Mathematikmodus nutzt
\usepackage{amsmath}
% amssymb liefert diverse Symbole für den Mathemodus
% unter anderem \mathbb für doppelt gestrichene Buchstaben (bb = blackboard)
\usepackage{amssymb}
% amsthm bietet Formatierung für Sätze, Bemerkungen, Lemmas, Aufgaben, ...
\usepackage{amsthm}
% für mehr Einstellungsmöglichkeiten gibt es zusätzlich \usepackage{ntheorem}

\newtheorem{thm}{Satz}[section]
% Gib Autor, Titel und Datum an
% wird bei \maketitle genutzt
\author{Felix Hilsky}
\title{Referenzierung}
\date{2017-05-23}

\begin{document}
  \maketitle

  \section{Einführung} \label{sec:einfuehrung}
  %
  Es gibt ein großes Problem, das hier beschrieben wird.
  %
  \section{Problem}
  %
  Wie in der Einführung~\ref{sec:einfuehrung} erwähnt, gibt es ein Problem. Es besteht darin... Das kann man auch in der Abbildung~\ref{fig:sinnvollesBild} erkennen.
  %
  \begin{figure}
    Hier könnte Ihr Bild stehen.
    \caption{Eine sinnvolle Bildunterschrift stehen.}
    \label{fig:sinnvollesBild}
  \end{figure}
  %
  \begin{thm}[Satz über tolle Graphen]\label{thm:tollerGraph}
    Ein Graph ist genau dann toll, wenn man ihn nicht mit endlich vielen Zeichen beschreiben kann:
    \begin{equation}
      G \text{ toll} \iff \nexists \{a_1, \dots, a_n\}: G = B(a_1, \dots, a_n)
      \label{equ:toll}
    \end{equation}
  \end{thm}
  %
  \section{Vorbereitungen}
  \subsection{Notation}
  Sei $G$ ein Graph. Wie wie in Satz~\ref{thm:tollerGraph} auf Seite~\pageref{thm:tollerGraph} gesehen haben, ist es schwer tolle Graphen anzugeben. Dabei ist $B$ wie in der Äquivalenz~\eqref{equ:toll} ein Operator, der natürliche Sätze interpretiert.
\end{document}