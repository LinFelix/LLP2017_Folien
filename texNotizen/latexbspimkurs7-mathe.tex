% Dokumentenklasse
% immer am Anfang des Dokuments
\documentclass{scrartcl}

% Präambel
\usepackage[utf8]{inputenc} % Eingabekodierung
\usepackage[T1]{fontenc}    % Schrifkodierung
\usepackage[ngerman]{babel} % optimiert für Deutsch (z.B. Titel, Silbentrennung)

\usepackage{amsmath}        % (nahezu) unerlässliches Paket für den Mathemodus
\usepackage{amssymb}        % ebenso
\title{Mathematikmodus}
\author{LLP-Kurs}
\date{nach Ostern}

% Ende Präambel
% Beginn eigentliches Dokument
\begin{document}
\maketitle
  Wir lernen jetzt den Mathematikmodus kennen. Er ist einer der wichtigsten Fähigkeiten von \TeX.

  Im Text kann man eine Formel $a + b = c$ mit Dollar-Zeichen schreiben. Aber auch mit \(a^2 + b^2 = dc_g^2\) den runden Klammern. Dagegen ist die Schreibweise
  \begin{math}
    \sum^5_{j=2} a_i = 7
  \end{math}
  etwas länger und daher nicht so beliebt.

  Größere Formeln sollten nicht im Text eingebettet werden. Dafür gibt es den abgesetzten Mathematik-Modus:
  \begin{equation}
    \frac{\alpha}{\gamma} = \varepsilon^2
  \end{equation}
  \begin{equation}
    \varepsilon < \aleph_0 - \aleph_{-1}
  \end{equation}

  Hier haben wir gerade zwei Formeln untereinander gehabt. Für diesen Zweck gibt es eine bessere Lösung, die das anordnen (align) ermöglicht:
  \begin{align}
    \sqrt[3]{7} &= 1,9129 \\
    \sqrt{7} &= 2.6457513110645907
  \end{align}
  Ohne das alignment sieht es nämlich so aus:
  \begin{align}
    \sqrt[3]{7} = 1,9129 \\
    \sqrt{7} = 2.6457513110645907
  \end{align}

  Um den Unterschied bei gleichem Inhalt zu sehen hier noch ein Beispiel:
  Dies $\sum_{i=1}^\infty\frac{1}{n}=\infty$ ist eine Textformel und
  \begin{equation}
    \sum_{i=1}^\infty\frac{1}{n} = \infty
  \end{equation}
  ist eine abgesetzte Formel.

  Wenn man sich diese Seite so anschaut, sieht es doch schon so aus, als hätte man etwas geschafft.

  Man kann alles ineinanderschachteln:
  \begin{equation}
    \gamma +
    \frac{
      \aleph^{
        \beth -
        \frac{
          1+\beta\cdot x^2
        }{
          \sqrt{2}
        }
      }-5
    }{
      \sqrt[2+\sqrt{\phi}]{
        42-\alpha-\theta-\mu
      }^\pi
    }
  \end{equation}
\end{document}
