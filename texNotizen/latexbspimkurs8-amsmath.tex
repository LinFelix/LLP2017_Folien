% Dokumentenklasse
% immer am Anfang des Dokuments
\documentclass{scrartcl}

% Präambel
\usepackage[utf8]{inputenc} % Eingabekodierung
\usepackage[T1]{fontenc}    % Schrifkodierung
\usepackage[ngerman]{babel} % optimiert für Deutsch (z.B. Titel, Silbentrennung)

\usepackage{amsmath}        % (nahezu) unerlässliches Paket für den Mathemodus
\usepackage{amssymb}        % ebenso
\title{Mathematikmodus}
\author{LLP-Kurs}
\date{nach Ostern}

% Ende Präambel
% Beginn eigentliches Dokument
\begin{document}
\maketitle
  Meistens kommt man für einzelne Gleichungen wie
  \begin{equation}
    \gamma +
    \frac{
      \aleph^{
        \beth -
        \frac{
          1+\beta\cdot x^2
        }{
          \sqrt{2}
        }
      }-5
    }{
      \sqrt[2+\sqrt{\phi}]{
        42-\alpha-\theta-\mu
      }^\pi
    }
  \end{equation}
  oder mehrzeiligen Gleichungen mit mehreren Umformungen
  \begin{align}
    (a+b)^2 &= ( a+b ) (a+b) & \text{Def. $\cdot^{\cdot}$} \\
            &= a (a+b) + b (a+b) & \text{Linksdistributivität} \\
            &= aa + ab + ba + bb & \text{Rechtsdistributivität} \\
            &= a^2 + ab + ba + b^2 & \text{Def. $\cdot^{\cdot}$} \\
            &= a^2 + 2 ab + b^2 & \text{Kommutativität}
  \end{align}
\end{document}
