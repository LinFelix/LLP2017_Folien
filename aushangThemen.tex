\documentclass[12pt]{scrartcl}

\usepackage{libertine}

\usepackage{graphicx}
\usepackage{hyperref}
% \usepackage{enumitem}

\usepackage[margin=2.3cm]{geometry}
\pagestyle{empty}
\newcommand{\chsep}{-2ex}

\renewcommand{\emph}{\textbf}
\begin{document}
  \centering
  \begin{figure}[t]
    \flushright
    \includegraphics[height=2.5cm]{../fsrlogo.png}
  \end{figure}
  % \renewcommand{\baselinestretch}{1.1}\normalsize
  {\Huge \textbf{
  Themen „Linux/\LaTeX/Python“}}
  \vspace{0.5cm}

  {\Large Ab Dienstag, 02.05.2017: \textbf{\LaTeX}
  \begin{itemize}%[itemsep=\chsep, topsep=\chsep, parsep=\chsep]
    \item Einführung
       \item Struktur, Formeln
       \item Tabellen, Graphiken
       \item Verweise, Bibliographie
    \item Formatierung
    \item Präsentationen
    \item Lua\LaTeX, Programmierung
  \end{itemize}
  % \setlength{\parskip}{-10pt}
  Ab etwa 2017-06-20: \textbf{Python}
  \begin{itemize}%[itemsep=\chsep, topsep=\chsep, parsep=\chsep]
    \item Einführung
    \item „Skripte für den Hausgebrauch“
    \item Mathematisches Rechnen
  \end{itemize}
  Wir passen uns euren Wünschen und eurem Bedarf an.

  Alle Informationen finden sich auf \url{myfsr.de/llp}.

  Wenn es euch möglich ist, bringt euren eigenen Laptop mit. Steht euch
keiner zur Verfügung, sprecht uns an und wir finden eine Lösung.

  Bitte installiert \LaTeX{} auf eurem Computer vor dem Kurs. Bei Problemen wendet euch an uns (\url{llp@myfsr.de}).}

  \vspace{2ex}
  
  % Die \renewcommand{\baselinestretch}{1}\normalsize
  -- Felix (Lin), Daniel und Felix (Bela)
\end{document}