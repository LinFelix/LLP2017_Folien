\documentclass[a4paper]{article}
\usepackage[T1]{fontenc}
\usepackage[utf8]{inputenx}
\usepackage{tikz}
\usepackage{qrcode}
\usetikzlibrary{decorations.pathmorphing,decorations.footprints,lindenmayersystems,shadows,decorations.fractals}
\newbox\mybox
\pagestyle{empty}
\usepackage{tgpagella,tgheros}
\usepackage[ngerman]{babel}
\usepackage{hyperref}

\begin{document}

\begin{tikzpicture}[overlay,shift={(5.5,-11)}] % yes, you *could* do it right, but well …

  \coordinate (front) at (0,0);
  \coordinate (horizon) at (0,.31\paperheight);
  \coordinate (bottom) at (0,-.6\paperheight);
  \coordinate (sky) at (0,.57\paperheight);
  \coordinate (left) at (-.51\paperwidth,0);
  \coordinate (right) at (.51\paperwidth,0);

  \shade [bottom color=white, top color=blue!30!black!50]
    ([yshift=-5mm]horizon -| left) rectangle (sky -| right);

  \shade [bottom color=black!70!green!25, top color=black!70!green!10]
    (front -| left) -- (horizon -| left)
    decorate [decoration=random steps] {-- (horizon -| right) }
    -- (front -| right) -- cycle;

  \shade [top color=black!70!green!25, bottom color=green!20!brown!50!white]
    ([yshift=-5mm-1pt]front -| left) rectangle ([yshift=1pt]front -| right);
  \fill [green!20!brown!50!white] (bottom -| left) rectangle ([yshift=-5mm]front -| right);

  \def\nodeshadowed[#1]#2;{
    \node[scale=2,above,anchor=base,#1]{
      \global\setbox\mybox=\hbox{#2}
      \copy\mybox};
    \node[scale=2,
          above,
          anchor=base,
          #1,
          yscale=-1,
          scope fading=south,
          opacity=0.2,
          yshift=0.5cm,
          xshift=-0.2cm]
      {\box\mybox};
  }

  \foreach \wherex/\wherey in {7cm/7cm,9cm/8cm,5cm/8.5cm,3cm/7.3cm,2cm/6.9cm,0cm/7.8cm} {
    \nodeshadowed[at={(\wherex,\wherey)}]{%
      \tikz \draw [green!20!black, rotate=90]
        l-system [l-system={
                     rule set={F -> FF-[-F+F]+[+F-F]},
                     axiom=F,
                     order=4,
                     step=2pt,
                     randomize step percent=50,
                     angle=30,
                     randomize angle percent=5}];};}

  \foreach \i in {0.5,0.6,...,2}
    \fill [white,
           opacity=\i/2,
           decoration=Koch snowflake,
           shift=(horizon),
           shift={(rand*11,rand*7)},
           scale=\i,
           double copy shadow={%
             opacity=0.2,
             shadow xshift=0pt,
             shadow yshift=3*\i pt,
             fill=white,
             draw=none}
           ]
      decorate {
        decorate {
          decorate {
            (0,0)-- ++(60:1) -- ++(-60:1) -- cycle
          }
        }
      };

  \foreach \x in {1,2,3}
    \fill [decorate,
           decoration={footprints,foot of=gnome},
           opacity=.5,
           red!30]
      (rand*8,-rand*10) to [out=rand*180,in=rand*180](rand*8,-rand*10);

  \nodeshadowed[at={(-6.0, 8.7)}, yslant=0.05]{\Huge\scalebox{1.5}{\textbf{\LaTeX}}};
  \nodeshadowed[at={(-0.5, 5.2)}, yslant=0.05]{\Huge\scalebox{1.5}{\textbf{Linux}}};
  \nodeshadowed[at={( 5.5, 1.7)}, yslant=0.05]{\Huge\scalebox{1.5}{\textbf{Python}}};

  \node[anchor=north west] at (-9,-1.5) {\LARGE \parbox{18cm}{\sf\bf%
      Ein Einführungskurs in das wissenschaftliche Arbeiten mit \LaTeX, GNU/Linux und
      Python.\\[-0.5cm]

      \begin{quote}
        \textsc{\color{blue!30!black}{Wo?}}\quad WIL/A120\\[-0.5cm]

        \textsc{\color{blue!30!black}Wann?}\quad Dienstag 6. DS, ab 11.\ April und am 4.\ April Linux-Installations-Party\\[-0.5cm]

        \textsc{\color{blue!30!black}Für Wen?}\quad Alle, die Interesse haben\\[-0.5cm]
      \end{quote}

      Laptop und USB-Stick sind von Vorteil, da praktische Übungen Bestandteil des Kurses sind.\\[-0.5cm]

      Weitere Informationen unter
      \begin{center}
        \textcolor{blue!30!black}{www.myfsr.de/LLP}\quad\qrcode %[height=4cm]
        {https://www.myfsr.de/dokuwiki/doku.php?id=veranstaltungen:llp}
      \end{center}
    }};

  \node[anchor=south west] at (-10.3,-13.9) {%
    \tiny{Dieses Poster enthält \LaTeX-Code aus dem PGF-Manual. Diesese Plakat basiert auf einem Plakat von Daniel, Maximilian und Tom, erstellt September 2014, quellen und source code unter https://lat.inf.tu-dresden.de/~borch/lehre/llp.html}};
\end{tikzpicture}

\end{document}

%%% Local Variables:
%%% mode: latex
%%% TeX-master: t
%%% ispell-local-dictionary: "de_DE"
%%% End:
